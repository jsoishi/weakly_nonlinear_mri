\documentclass[letterpaper,12pt]{article}

%% Page dimensions for 1 inch margins
\setlength{\textwidth}{6.5in}
\setlength{\textheight}{9in}
\setlength{\topmargin}{-0in}
\setlength{\oddsidemargin}{0in}
\setlength{\evensidemargin}{0in} 
\setlength{\headheight}{0in}
\setlength{\headsep}{0in} 
\setlength{\hoffset}{0in}
\setlength{\voffset}{0in}

%% Graphics and math
\usepackage{graphics,graphicx,subfigure}
\usepackage{amsmath}

\newcommand\lsim{\mathrel{\rlap{\lower4pt\hbox{\hskip1pt$\sim$}}
        \raise1pt\hbox{$<$}}}
\newcommand\gsim{\mathrel{\rlap{\lower4pt\hbox{\hskip1pt$\sim$}}
        \raise1pt\hbox{$>$}}}

%% Bibliographic stuff

%\usepackage[numbers]{natbib}
\usepackage{natbib}
\bibliographystyle{apj}
%\setlength{\bibsep}{0.0pt}

\newcommand\apjl{ApJL}
\newcommand\apj{ApJ}
\newcommand\aj{AJ}
\newcommand\apjs{ApJS}
\newcommand\aap{AAP}
\newcommand\mnras{MNRAS}
\newcommand\araa{ ARA\&A }
\newcommand\nat{ Nature }

%Use this for citing without parentheses.
\newcommand{\citei}[1]{\citeauthor{#1} \citeyear{#1}}

\begin{document}
\pagestyle{plain}
\pagenumbering{arabic}

\title{Weakly Nonlinear MRI: Notes}
\author{Susan Clark}
\maketitle
\vspace{0.2cm}

The basic setup for this investigation will follow \citet{Umurhan:2007dz}, who apply a weakly nonlinear analysis to the MRI in the shearing box approximation, in a thin-gap limit with idealized boundary conditions. We will follow a similar derivation, without the thin-gap restriction and with more realistic boundary conditions.

The shearing box (SB) approximation is one in which one considers the flow in a small box at some fiducial distance ${r_0}$ away from the center of the rotating system. This constitutes only a small portion of an accretion disk, but makes the problem more tractable, both for analytic and numerical analyses. 

My understanding is that the thin-gap limit allows one to ignore radially-dependent (``channel") modes. Channel modes are axially symmetric, linear modes, and are exact solutions of the nonlinear MRI in the SB approximation. These are the so-called ``primary MRI" modes discussed in \citet{Pessah:2010ic}. The advantage to analyzing the MRI without channel modes is that the background is steady and has a constant pressure. By widening our gap, we will complicate things because the non-constant background means that we open ourselves to parasitic modes -- modes whose growth rates depend on the amplitudes of the primary modes. In particular, the channel modes are Kelvin-Helmholtz unstable. 

Another way to say this is that the wide gap approximation does not allow us to ignore the $\left(\textbf{u} \cdot \nabla \right) \textbf{u}$ term in the Navier-Stokes equation, which couples velocities with each other and thus allows the existence of parasitic modes. This can be seen by considering $\textbf{u}$ as a sum of discrete modes with wavenumber k:

\begin{equation}
\textbf{u} = \sum\limits_{k = 0}^{\infty}a_k e^{ikx} \, dk
\end{equation}

\noindent The Fourier transform of $\left(\textbf{u} \cdot \nabla \right) \textbf{u}$ is:

\begin{equation}
a_k \otimes i \textbf{k} \cdot a_k
\end{equation}

\noindent Clearly, each wavenumber (and its associated amplitude) depends on all other wave numbers.\\

So essentially, our approach is similar to \citet{Umurhan:2007dz}, but our setup is complementary to \citet{Pessah:2010ic}. \\

\noindent The MRI setup in \citet{Umurhan:2007dz} is as follows: 
\begin{itemize}
\item An axisymmetric channel flow with a differential rotation ${\Omega\left(r\right) \propto \Omega_0\left(\frac{r}{r_0}\right)^{-q}}$ 
\item A velocity $\textbf{V}$ = $U\left(x\right)\hat{\bf{y}}$ with a linear shear profile $U\left(x\right) = -qU_0x$
\item Constant, vertical magnetic field $\bf{\hat{B}}$ = $B_0\hat{z}$
\end{itemize}

We will use the incompressible limit ($\nabla \cdot \textbf{u}$ = $0$), which applies because we consider weak magnetic fields (magnetic energy is small compared to thermal energy).



\bibliography{mri.bib}

\end{document}