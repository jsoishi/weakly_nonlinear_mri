\documentclass[letterpaper,12pt]{article}

%% Page dimensions for 1 inch margins
\setlength{\textwidth}{6.5in}
\setlength{\textheight}{9in}
\setlength{\topmargin}{-0in}
\setlength{\oddsidemargin}{0in}
\setlength{\evensidemargin}{0in} 
\setlength{\headheight}{0in}
\setlength{\headsep}{0in} 
\setlength{\hoffset}{0in}
\setlength{\voffset}{0in}

%% Graphics and math
\usepackage{graphics,graphicx,subfigure}
\usepackage{amsmath}

\newcommand\lsim{\mathrel{\rlap{\lower4pt\hbox{\hskip1pt$\sim$}}
        \raise1pt\hbox{$<$}}}
\newcommand\gsim{\mathrel{\rlap{\lower4pt\hbox{\hskip1pt$\sim$}}
        \raise1pt\hbox{$>$}}}

%% Bibliographic stuff

%\usepackage[numbers]{natbib}
\usepackage{natbib}
\bibliographystyle{apj}
%\setlength{\bibsep}{0.0pt}

\newcommand\apjl{ApJL}
\newcommand\apj{ApJ}
\newcommand\aj{AJ}
\newcommand\apjs{ApJS}
\newcommand\aap{AAP}
\newcommand\mnras{MNRAS}
\newcommand\araa{ ARA\&A }
\newcommand\nat{ Nature }

%Use this for citing without parentheses.
\newcommand{\citei}[1]{\citeauthor{#1} \citeyear{#1}}

\begin{document}
\pagestyle{plain}
\pagenumbering{arabic}

\title{Weakly Nonlinear MRI: Notes}
\author{Susan Clark}
\maketitle
\vspace{0.2cm}

The basic setup for this investigation will follow \citet{Umurhan:2007dz}, who apply a weakly nonlinear analysis to the MRI in the shearing box approximation, in a thin-gap limit with idealized boundary conditions. We will follow a similar derivation, without the thin-gap restriction and with more realistic boundary conditions.

The shearing box (SB) approximation is one in which one considers the flow in a small box at some fiducial distance ${r_0}$ away from the center of the rotating system. This constitutes only a small portion of an accretion disk, but makes the problem more tractable, both for analytic and numerical analyses. 

My understanding is that the thin-gap limit allows one to ignore radially-dependent (``channel") modes. Channel modes are axially symmetric, linear modes, and are exact solutions of the nonlinear MRI in the SB approximation. These are the so-called ``primary MRI" modes discussed in \citet{Pessah:2010ic}. The advantage to analyzing the MRI without channel modes is that the background is steady and has a constant pressure. By widening our gap, we will complicate things because the non-constant background means that we open ourselves to parasitic modes -- modes whose growth rates depend on the amplitudes of the primary modes. In particular, the channel modes are Kelvin-Helmholtz unstable. 

%Another way to say this is that the wide gap approximation does not allow us to
In general, we cannot ignore the $\left(\textbf{u} \cdot \nabla \right) \textbf{u}$ term in the Navier-Stokes equation, which couples velocities with each other. %and thus allows the existence of parasitic modes. 
This can be seen by considering $\textbf{u}$ as a sum of discrete modes with wavenumber k:

\begin{equation}
\textbf{u} = \sum\limits_{k = 0}^{\infty}a_k e^{ikx} \, dk
\end{equation}

\noindent The Fourier transform of $\left(\textbf{u} \cdot \nabla \right) \textbf{u}$ is:

\begin{equation}
a_k \otimes i \textbf{k} \cdot a_k
\end{equation}

\noindent Clearly, each wavenumber (and its associated amplitude) depends on all other wave numbers.\\

So essentially, our approach is similar to \citet{Umurhan:2007dz}, but our setup is complementary to \citet{Pessah:2010ic}. \\

\noindent The MRI setup in \citet{Umurhan:2007dz} is as follows: 
\begin{itemize}
\item An axisymmetric channel flow with a differential rotation ${\Omega\left(r\right) \propto \Omega_0\left(\frac{r}{r_0}\right)^{-q}}$ 
\item A velocity $\textbf{V}$ = $U\left(x\right)\hat{\bf{y}}$ with a linear shear profile $U\left(x\right) = -qU_0x$
\item Constant, vertical magnetic field $\bf{\hat{B}}$ = $B_0\hat{z}$
\end{itemize}

We will use the incompressible limit ($\nabla \cdot \textbf{u}$ = $0$), which applies because we consider weak magnetic fields (magnetic energy is small compared to thermal energy). \\

\noindent \textbf{We will proceed as follows:}

\begin{itemize}
\item Linearize equations for perturbations of the form $e^{st + ik_x x + ik_z z}$. We neglect y-dependence because of axisymmetry.
\item Write linearized equations in matrix form, use SymPy to obtain dispersion relation. This is a 4th order polynomial in s.
\end{itemize}
\textit{The following five bullet points are performed in analogy to Raleigh B\'enard convection:}
\begin{itemize}
\item We want to derive an envelope (amplitude) equation, which we will do with a multiscale analysis. We choose scalings such that $s$, $k_x$, and $k_z$ in our dispersion relation come in at the same (lowest) order in the small parameter $\epsilon$. This determines the scalings for the corresponding slow variables T, X, and Y.
\item Rewrite equations in terms of quantities that depend on both fast and slow variables, in the form $u(x, y, t; X, Y, T) = A(X, Y, T)e^{i\mathbf{k} \cdot \mathbf{x}}$.
\item Expand quantities in powers of $\epsilon$.
\item Group terms by $\mathcal{O}\left(\epsilon\right)$
\item Solve successively to obtain envelope equation. We solve up to the order where a term appears that can cause saturation.\\

%\item We are only interested in marginality to the MRI, so we will take s=0. 
%\item This should allow us to isolate the most unstable mode at criticality. 


\end{itemize} 

\noindent Things to include:
\begin{itemize}
\item Derivations that I wrote out by hand (scan)
\item Nondimensionalization 
\item Multiscale analysis (in analogy to Raleigh B\'enard convection)
\item Boundary conditions
\item Amplitude functions
\item GLE
\end{itemize}

\vspace{10cm}

\textbf{Derivation of Equations} \\

Begin with the Navier-Stokes equation:

\begin{equation}
\partial_t \mathbf{u} \, + \, \mathbf{u} \cdot \nabla \mathbf{u} \, = \, -\frac{1}{\rho}\nabla P \, - \, \nabla\Phi \, + \, \frac{1}{\rho} \left(\mathbf{J}\times\mathbf{B}\right) \, - \, \nu\nabla^2 \mathbf{u} \, - \, 2\mathbf{\Omega} \times \mathbf{u} \, - \, \mathbf{\Omega} \times \left(\mathbf{\Omega} \times \mathbf{r} \right)
\end{equation}\label{momeqn}

The reference frame is rotating with angular velocity $\Omega_0 \mathbf{\hat{z}}$. \\

The coriolis term becomes: \\

$2\mathbf{\Omega} \times \mathbf{u} \, = \, 2\Omega_0 \mathbf{\hat{z}} \times \mathbf{u} \, = \, 2\Omega_0\left(\mathbf{\hat{z}} \times u_r \mathbf{\hat{r}} \, + \, \mathbf{\hat{z}} \times u_{\phi} \mathbf{\hat{\phi}}\right) \, = \, 2\Omega_0\left(v_r \mathbf{\hat{\phi}} - v_{\phi}\mathbf{\hat{r}}\right) $\\

The centrifugal term becomes: \\

$\mathbf{\Omega} \times \left(\mathbf{\Omega} \times \mathbf{r} \right) \, = \, \Omega_0 \mathbf{\hat{z}} \times \left(\Omega_0 \mathbf{\hat{z}} \times \mathbf{\hat{r}}\right) = \Omega_0^2 \mathbf{\hat{z}} \times \left(r \mathbf{\hat{\phi}}\right) \, = \, -\Omega_0^2 r \mathbf{\hat{r}}$ \\

The gravitational force per unit mass can be written: \\

$\nabla\Phi \, = \, -\frac{1}{\rho_b}\nabla P_b + r\Omega^2 \mathbf{\hat{r}}$ \\

Where $\rho_b$ and $P_b$ are the density and pressure that satisfy a barotropic relation in a steady axisymmetric solution. \\

The current density is $\mathbf{J} = \frac{1}{4\pi} \nabla \times \mathbf{B}$ and thus the Lorentz force becomes: \\

$\frac{1}{\rho} \left(\mathbf{J}\times\mathbf{B}\right) = \frac{1}{4 \pi \rho} \left( \nabla \times \mathbf{B} \right) \times \mathbf{B} = - \frac{1}{8 \pi \rho}\nabla B^2 + \frac{1}{4 \pi \rho} \left(\mathbf{B} \cdot \nabla\right)\mathbf{B}$ \\

Using the identity: \\

$\left( \nabla \times \mathbf{B} \right) \times \mathbf{B} = - \mathbf{B} \times \left( \nabla \times \mathbf{B} \right) = -\nabla \left(\mathbf{B} \cdot \mathbf{B}\right) + \left(\mathbf{B} \cdot \nabla \right)\mathbf{B} + \left(\mathbf{B} \cdot \nabla \right)\mathbf{B} + \mathbf{B} \times \left(\nabla \times \mathbf{B}\right)$ \\

The initial magnetic field is a constant vertical field $\mathbf{B} = B_0 \mathbf{\hat{z}}$, and thus the magnetic pressure term becomes: \\

$- \frac{1}{8 \pi \rho}\nabla B^2 \,= \, - \frac{1}{8 \pi \rho} \left(2 B_0 \partial_z \mathbf{B}\right) \, = \, - \frac{1}{4 \pi \rho} \left(B_0 \partial_z \mathbf{B}\right)$ \\

We now have: 

\begin{equation}
\partial_t \mathbf{u} \, + \, \mathbf{u} \cdot \nabla \mathbf{u} \, + 2\Omega_0\left(v_r \mathbf{\hat{\phi}} - v_{\phi}\mathbf{\hat{r}}\right) \, = \, - r\left(\Omega^2 - \Omega_0^2\right)\mathbf{\hat{r}} \, -\frac{1}{\rho}\nabla P \, +\frac{1}{\rho_b}\nabla P_b \,  - \, \frac{1}{4 \pi \rho} \left(B_0 \partial_z \mathbf{B}\right) + \frac{1}{4 \pi \rho} \left(\mathbf{B} \cdot \nabla\right)\mathbf{B}  \, - \, \nu\nabla^2 \mathbf{u} \,
\end{equation}

The net radial body force in the rotating frame, $r\left(\Omega^2 - \Omega_0^2\right)$, is Taylor expanded as follows: \\

$-r\left(\Omega^2 - \Omega_0^2\right) \,= \,-2 r_0 \Omega\left(r_0\right)\left(r - r_0\right)\left.\frac{\partial \Omega}{\partial r}\right|_{r_0} \,= \, -2 r_0 \Omega\left(r_0\right)\left(r - r_0\right)\frac{\Omega_0}{r_0} \left.\frac{\partial ln \Omega}{\partial ln r}\right|_{r_0} $ \\

And the definition -$\left.\frac{\partial ln \Omega}{\partial ln r}\right|_{r_0} \equiv q$ is introduced. \\

We now finally make the switch 

\bibliography{mri.bib}

\end{document}