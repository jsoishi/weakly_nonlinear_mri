\documentclass[11p]{paper}

%%%%%%begin preamble
%\usepackage[hmargin=1in, vmargin=1in]{geometry} % Margins
\usepackage{url}
\usepackage{hyperref}
%\usepackage{times}
\usepackage{natbib}
\usepackage{graphicx}
\usepackage{amsmath}
\usepackage{amsfonts}
\usepackage{amssymb}
\usepackage{pdfpages}

%\usepackage{fontspec}
%\setmainfont{TimesNewRoman}
%\definetypeface[Serapion][rm][Xserif][Serapion Pro]
%\setupbodyfont[Serapion, 12pt]

%%%
%%%%%% uncomment following 4 lines to adjust title size/shape and
%%%%%% trailing space
%% \usepackage{titling}
%% %\pretitle{\noindent\Large\bfseries}
%% \date{}
%% \setlength{\droptitle}{-1in}
%\posttitle{\\}

\setcounter{tocdepth}{2}
%% headers
\usepackage{fancyhdr}
\pagestyle{fancy}
\lhead{}
\chead{}
\rhead{Weakly Nonlinear MRI Notes}
\lfoot{}
\cfoot{\thepage}
\rfoot{}

\newcommand{\Reyn}{\ensuremath{\mathrm{Re}}}
\newcommand{\Rmag}{\ensuremath{\mathrm{Rm}}}
\newcommand{\Rmagc}{\ensuremath{\mathrm{Rm}_\mathrm{crit}}}
\newcommand{\Prandtl}{\ensuremath{\mathrm{Pm}}}
\newcommand{\Lund}{\ensuremath{\mathrm{S}}}
\newcommand{\Lundc}{\ensuremath{\mathrm{S}_\mathrm{crit}}}
\newcommand{\yt}{\texttt{yt}}
\newcommand{\enzo}{\texttt{Enzo}}
\newcommand{\nosection}[1]{%
  \refstepcounter{section}%
  \addcontentsline{toc}{section}{\protect\numberline{\thesection}#1}%
  \markright{#1}}
\newcommand{\shellcmd}[1]{\\\indent\indent\texttt{\footnotesize\$ #1}\\}

%%%%%%end preamble

\title{Some Notes on the Weakly Nonlinear MRI Project}
\begin{document}
\maketitle

\section{Overview of The Problem}
\label{sec:overview}

We want to construct an analytical understanding of the saturation of
the MRI, complementary to the linear analysis of \citet{Pessah2010}.

In order to understand saturation of the MRI, we need to consider the
options available to it. 

\section{Mathematical Basics}
\label{sec:basics}

Here are some really basic math stuff that might be useful to
you. First some notation. You'll see the partial derivative of a
function $f$ with respect to $x$,
\begin{equation*}
  \label{eq:partial}
  \frac{\partial f}{\partial x}
\end{equation*}
written as $\partial_x f$, or even $f_x$, especially in the applied
math literature. I tend to use $\partial_x f$, because it's terse but
not so terse I forget what it is I meant by it.

In order to solve the equations of magnetohydrodyanmics (MHD), we'll
need names for the velocity, density, magnetic, and pressure
fields. Typically, the velocity vector is written as $\mathbf{v}$ or
$\mathbf{u}$ (I tend to use the latter). The components of
$\mathbf{u}$ are written either as $(u_x, u_y, u_z)$ (typical in the
astrophysical literature) or $(u, v, w)$ (more common in geophysical
fluid dynamics). Density is nearly always $\rho$. The magnetic field
is usually $\mathbf{B}$ when a vector, with scalar components $(B_x,
B_y, B_z)$ though you may (very rarely) see $(a, b, c)$. Obviously,
you'd substitute the subscripts for $r, \theta, \phi$ in spherical and
$r, \theta, z$ in cylindrical coordinates. Occasionally you will see
the cylindrical radius written $\varpi$.

In compressible gas dynamics, the sound speed is usually written as
$c$ or $c_s$. There rarely is confusion with the speed of light, even
in relativistic (special or general) MHD.

\subsection{The Equations of MHD}
\label{sec:equations}

There are a lot of ways to write the equations we need, and they go by
a lot of names. We will start by writing the equations of
hydrodynamics, and then add MHD later. Fundamentally, these equations
express the conservation of mass, momentum, and energy. For MHD, we
also need to describe the evolution of the magnetic field. Please note
that we are not saying anything physical about fluids, plasmas, or
(magneto)hydrodynamics here. We're just laying out the relevant
equations. There is a lot to say about MHD, but I will only say one
physical thing here: it is really, really important to remember that
MHD equations are not like general relativity or Maxwell's equations
or the Schrodinger Equation: they are themselves an
approximation. They are \emph{not} always valid: the description of a
system as a continuum of neutral (hydrodynamics) or charged (MHD)
particles can break down without the mathematical solution you've
painstakingly derived telling you a thing about it. This is especially
important to keep in mind because this whole project is about
constructing analytic solutions by making loads of approximations.

For a fluid, mass conservation is given by 
\begin{equation}
  \label{eq:continuity}
  \partial_t \rho + \nabla \cdot \rho \mathbf{u} = 0.
\end{equation}
This is also known as the \textit{continuity} equation, because it not
only says that mass is conserved, it also states that mass must flow
continuously from one place to another. In this form, it is the model
of a class of equations called the conservation laws, because you can
think of it as saying the following. The time rate of change of the
mass of fluid within a given volume $\mathcal{V}$ is given by the
\emph{flux} $\mathcal{F} = \rho \mathbf{u}$ across the surface
bounding $\mathcal{V}$.

\begin{equation}
  \label{eq:navier-stokes}
  \partial_t \mathbf{u} + \mathbf{u \cdot \nabla u} = -\frac{\nabla p}{\rho} + \nu \nabla^2 \mathbf{u},
\end{equation}
which is also called the \textit{Navier-Stokes} equation when $\nu
\neq 0$ and the \textit{Euler} equation otherwise. This equation is
essentially $F = ma$ for a fluid: any additional physics we want to
add will add force terms to the right hand side of this equation. Most
important for us will be the fictitious rotational forces and the
Lorentz force of MHD. 

Of course, we also need an equation for the conservation of energy (that is, we must satisfy the First Law of Thermodynamics). However, we will be working exclusively with \textit{incompressible} fluids, so we can simply enforce the \emph{incompressibility constraint}, 
\begin{equation}
  \label{eq:incompressibility}
  \mathbf{\nabla \cdot u} = 0.
\end{equation}
This is called a ``constraint'' equation because it does not specify
the time evolution of the system, but instead specifies a certain
property the solution must obey. 

To add the magneto to hydrodynamics, we first introduce the Lorentz
force per unit volume,
\begin{equation}
  \label{eq:lorentz}
  \mathbf{f_L = \frac{J \times B}{\rho}}.
\end{equation}
The current density (current per unit volume) is just
\begin{equation}
  \label{eq:current}
  \mathbf{J = \frac{\nabla \times B}{4\pi}}
\end{equation}
in Gaussian units (which we'll use. Sorry.). We then combine the $f_L$
with the definition of $J$, and using a vector calculus you can easily
look up, we arrive at 
\begin{equation}
  \label{eq:lorentz_final}
  \mathbf{F_L = -\frac{\nabla B^2}{8 \pi \rho} + \frac{B \cdot \nabla B}{4
      \pi \rho}}.
\end{equation}
This form is very telling, because it splits the Lorentz force into
two pieces: the \emph{magnetic pressure} ($\nabla B^2/8 \pi$) and the
\emph{magnetic tension} ($B \cdot \nabla B / 4 \pi$). The physics of
the MRI is contained in the latter: magnetic fields, when stretched
act exactly like elastic bands. 

Of course, now we have added a new vector field, $\mathbf{B}$, to our
equation set, so obviously we need three more equations

I'm glossing over a lot here, but for MHD, where the plasma behaves
like a fluid, the ``Ohm's Law'', or relationship between the Electric
Field and the current, gives
\begin{equation}
  \label{eq:ohms}
  \mathbf{E} = -\frac{\mathbf{u \times B}}{c} - \frac{\mathbf{J}}{\sigma},
\end{equation}
where $\sigma$ is the \emph{conductivity} of the plasma (we typically
never talk in terms of the conductivity, preferring the resistivity,
which is implicitly derived below).

Now we can employ one of the Maxwell's equations,
\begin{equation}
  \label{eq:maxwell}
  \partial_t \mathbf{B} = -\frac{c}{4\pi} \mathbf{\nabla \times E},
\end{equation}
substituting in equation~(\ref{eq:ohms}), we arrive at the
\emph{induction} equation, sometimes also known as the \emph{MHD}
equation,
\begin{equation}
  \label{eq:induction}
  \partial_t \mathbf{B = \nabla \times (u \times B) + \mu \nabla^2 B},
\end{equation}
where $\mu$ is the \emph{resistivity} of the plasma.
\section{Computational Basics}
\label{sec:comp_basics}

We'll be making a lot of use of Python, SymPy, the iPython notebook,
and later, Dedalus 2.0. 

\subsection{Software to install}
\label{sec:software}

\begin{itemize}
\item Python3 (latest version)
\item SymPy 
\item ipython 1.0
\item matplotlib 1.3.0
\item mercurial (hg)
\end{itemize}

I know you use a Mac, which I don't, so I'm not an expert. If you
haven't already, one way to do this is to install python3, and then
install pip using the \texttt{get-pip.py} script described at
\url{http://www.pip-installer.org/en/latest/installing.html}. Once you have pip installed, you can just do 
\shellcmd{pip install sympy} 
for example, and everything should just work.

\subsection{Dedalus 2.0}
\label{sec:dedalus2}

Dedalus 2.0 isn't quite ready for us to use yet, and it's very likely
a bunch of stuff will change between now and when we are ready, but if
you're curious, it's here: \url{https://bitbucket.org/jsoishi/dedalus2}.

\subsection{Version Control with Mercurial and Bitbucket}
\label{sec:vc}

In order to facilitate our collaboration, we'll be using the mercurial
version control system. A great intro to is
\url{http://hginit.com}. 

Our shared repository will be hosted on
\url{https://bitbucket.org}. You'll need to first setup a bitbucket
account. Just go to the site, create a user name, and email it to
me. I'll add you to the access list for our, which you can find here:
\url{https://bitbucket.org/jsoishi/weakly_nonlinear_MRI/}. Until
you've created a user name and emailed me, I'll leave it public so you
can see how it works without having to wait for me. However, after
you've gotten your user name, we'll make it private, so you don't have
to worry about anyone seeing our mistakes (which we will make plenty
of!). Of course, there is tremendous value in open science, so I
typically open up my repositories after publication. I will encourage
but not require you to do the same.

\section{Roadmap}
\label{sec:roadmap}

Here's a very rough, and sure to change outline of the project as I
see it now. 

\begin{enumerate}
\item Background reading
\item Linearize the MHD equations in a shearing box
\item Derive the linear conditions for the MRI with viscosity and resistivity
\item Derive the Ginzburg-Landau equation from convection
\item Derive the Ginzburg-Landau equation from the MHD equations in a
  thin-gap Taylor-Couette flow (rederive \citet{2007PhRvL..98c4501U})
\item Extend weakly nonlinear analysis to the wide gap case
\end{enumerate}

Once we get down to the bottom of the list, we can begin to understand

\section{Annotated Bibliography}
\label{sec:annotated_bib}

Basics on the CGLE: ``The complex Ginzburg-Landau equation: an introduction,'' Garcia-Morales, V \& Krischner, K. \url{http://www.tandfonline.com/doi/pdf/10.1080/00107514.2011.642554}

\bibliographystyle{apj}
\bibliography{library}
\end{document}
%  LocalWords:  Pessah magnetohydrodyanmics MHD Navier SymPy iPython
%  LocalWords:  Dedalus ipython matplotlib hg py sympy Bitbucket CGLE
%  LocalWords:  bitbucket Roadmap Linearize Ginzburg Couette
%  LocalWords:  Krischner
