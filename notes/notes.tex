\documentclass[11p]{paper}

%%%%%%begin preamble
\usepackage[hmargin=1in, vmargin=1in]{geometry} % Margins
\usepackage{url}
\usepackage{hyperref}
%\usepackage{times}
\usepackage{natbib}
\usepackage{graphicx}
\usepackage{amsmath}
\usepackage{amsfonts}
\usepackage{amssymb}
\usepackage{pdfpages}

%\usepackage{fontspec}
%\setmainfont{TimesNewRoman}
%\definetypeface[Serapion][rm][Xserif][Serapion Pro]
%\setupbodyfont[Serapion, 12pt]

%%%
%%%%%% uncomment following 4 lines to adjust title size/shape and
%%%%%% trailing space
%% \usepackage{titling}
%% %\pretitle{\noindent\Large\bfseries}
%% \date{}
%% \setlength{\droptitle}{-1in}
%\posttitle{\\}

\setcounter{tocdepth}{2}
%% headers
\usepackage{fancyhdr}
\pagestyle{fancy}
\lhead{}
\chead{}
\rhead{Weakly Nonlinear MRI Notes}
\lfoot{}
\cfoot{\thepage}
\rfoot{}

\newcommand{\Reyn}{\ensuremath{\mathrm{Re}}}
\newcommand{\Rmag}{\ensuremath{\mathrm{Rm}}}
\newcommand{\Rmagc}{\ensuremath{\mathrm{Rm}_\mathrm{crit}}}
\newcommand{\Prandtl}{\ensuremath{\mathrm{Pm}}}
\newcommand{\Lund}{\ensuremath{\mathrm{S}}}
\newcommand{\Lundc}{\ensuremath{\mathrm{S}_\mathrm{crit}}}
\newcommand{\yt}{\texttt{yt}}
\newcommand{\enzo}{\texttt{Enzo}}
\newcommand{\nosection}[1]{%
  \refstepcounter{section}%
  \addcontentsline{toc}{section}{\protect\numberline{\thesection}#1}%
  \markright{#1}}
\newcommand{\shellcmd}[1]{\\\indent\indent\texttt{\footnotesize\$ #1}\\}

%%%%%%end preamble

\title{Some Notes on the Weakly Nonlinear MRI Project}
\begin{document}
\maketitle

\section{Overview of The Problem}
\label{sec:overview}

We want to construct an analytical understanding of the saturation of
the MRI, complementary to the linear analysis of \citet{Pessah2010}.

In order to understand saturation of the MRI, we need to consider the
options available to it. 

\section{Mathematical Basics}
\label{sec:basics}

Here are some really basic math stuff that might be useful to
you. First some notation. Derivatives are a fun one. You'll see the
partial derivative of a function $f$ with respect to $x$, 
\begin{equation*}
  \label{eq:partial}
  \frac{\partial f}{\partial x}
\end{equation*}
written as $\partial_x f$, or even $f_x$, especially in the applied
math literature. I tend to use $\partial_x f$, because it's terse but
not so terse I forget what it is I meant by it.

In order to solve the equations of magnetohydrodyanmics (MHD), we'll
need names for the velocity, density, magnetic, and pressure
fields. Typically, the velocity vector is written as $\mathbf{v}$ or
$\mathbf{u}$ (I tend to use the latter). The components of
$\mathbf{u}$ are written either as $(u_x, u_y, u_z)$ (typical in the
astrophysical literature) or $(u, v, w)$ (more common in geophysical
fluid dynamics). Density is nearly always $\rho$. The magnetic field
is usually $\mathbf{B}$ when a vector, with scalar components $(B_x,
B_y, B_z)$ though you may (very rarely) see $(a, b, c)$. Obviously,
you'd substitute the subscripts for $r, \theta, \phi$ in spherical and
$r, \theta, z$ in cylindrical coordinates. Occasionally you will see
the cylindrical radius written $\altrho$.

In compressible gas dynamics, the sound speed is usually written as
$c$ or $c_s$. There rarely is confusion with the speed of light, even
in relativisitic (special or general) MHD. Of course, GR MHD is way
beyond the scope of this project, but it brings from GR a lot of very
interesting and rather different notation.

There are a lot of ways to write the equations we need, and they go by
a lot of names.

Fundamentally, they are

\begin{equation}
  \label{eq:continuity}
  \partial \rho
\end{equation}

\section{Computational Basics}
\label{sec:comp_basics}

We'll be making a lot of use of Python, SymPy, the iPython notebook,
and later, Dedalus 2.0. 

\subsection{Software to install}
\label{sec:software}

\begin{itemize}
\item Python3 (latest version)
\item SymPy 
\item ipython 1.0
\item matplotlib 1.3.0
\item mercurial (hg)
\end{itemize}

I know you use a Mac, which I don't, so I'm not an expert. If you
haven't already, one way to do this is to install python3, and then
install pip using the \texttt{get-pip.py} script described at
\url{http://www.pip-installer.org/en/latest/installing.html}. Once you have pip installed, you can just do 
\shellcmd{pip install sympy} 
for example, and everything should just work.

\subsection{Dedalus 2.0}
\label{sec:dedalus2}

Dedalus 2.0 isn't quite ready for us to use yet, and it's very likely
a bunch of stuff will change between now and when we are ready, but if
you're curious, it's here: \url{https://bitbucket.org/jsoishi/dedalus2}.

\subsection{Version Control with Mercurial and Bitbucket}
\label{sec:vc}

In order to facilitate our collaboration, we'll be using the mercurial
version control system. A great intro to is
\url{http://hginit.com}. 

Our shared repository will be hosted on
\url{https://bitbucket.org}. You'll need to first setup a bitbucket
account. Just go to the site, create a username, and email it to
me. I'll add you to the access list for our, which you can find here:
\url{https://bitbucket.org/jsoishi/weakly_nonlinear_MRI/}. Until
you've created a username and emailed me, I'll leave it public so you
can see how it works without having to wait for me. However, after
you've gotten your username, we'll make it private, so you don't have
to worry about anyone seeing our mistakes (which we will make plenty
of!). Of course, there is tremendous value in open science, so I
typically open up my repositories after publication. I will encourage
but not require you to do the same.

\section{Roadmap}
\label{sec:roadmap}

Here's a very rough, and sure to change outline of the project as I
see it now. 

\begin{enumerate}
\item Background reading
\item Linearize the MHD equations in a shearing box
\item Derive the linear conditions for the MRI with viscosity and resistivity
\item Derive the Ginzburg-Landau equation from convection
\item Derive the Ginzburg-Landau equation from the MHD equations in a
  thin-gap Taylor-Couette flow (rederive \citet{2007PhRvL..98c4501U})
\item Extend weakly nonlinear analysis to the wide gap case
\end{enumerate}

Once we get down to the bottom of the list, we can begin to understand

\section{Annotated Bibliography}
\label{sec:annotated_bib}

Basics on the CGLE: ``The complex Ginzburg-Landau equation: an introduction,'' Garcia-Morales, V \& Krischner, K. \url{http://www.tandfonline.com/doi/pdf/10.1080/00107514.2011.642554}

\bibliographystyle{apj}
\bibliography{library}
\end{document}