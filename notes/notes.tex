\documentclass[11p]{paper}

%%%%%%begin preamble
\usepackage[hmargin=1in, vmargin=1in]{geometry} % Margins
\usepackage{url}
\usepackage{hyperref}
\usepackage{times}
\usepackage{natbib}
\usepackage{graphicx}
\usepackage{amsmath}
\usepackage{amsfonts}
\usepackage{amssymb}
\usepackage{pdfpages}

%\usepackage{fontspec}
%\setmainfont{TimesNewRoman}
%\definetypeface[Serapion][rm][Xserif][Serapion Pro]
%\setupbodyfont[Serapion, 12pt]

%%%
%%%%%% uncomment following 4 lines to adjust title size/shape and
%%%%%% trailing space
%% \usepackage{titling}
%% %\pretitle{\noindent\Large\bfseries}
%% \date{}
%% \setlength{\droptitle}{-1in}
%\posttitle{\\}

\setcounter{tocdepth}{2}
%% headers
\usepackage{fancyhdr}
\pagestyle{fancy}
\lhead{}
\chead{}
\rhead{Weakly Nonlinear MRI Notes}
\lfoot{}
\cfoot{\thepage}
\rfoot{}

\newcommand{\Reyn}{\ensuremath{\mathrm{Re}}}
\newcommand{\Rmag}{\ensuremath{\mathrm{Rm}}}
\newcommand{\Rmagc}{\ensuremath{\mathrm{Rm}_\mathrm{crit}}}
\newcommand{\Prandtl}{\ensuremath{\mathrm{Pm}}}
\newcommand{\Lund}{\ensuremath{\mathrm{S}}}
\newcommand{\Lundc}{\ensuremath{\mathrm{S}_\mathrm{crit}}}
\newcommand{\yt}{\texttt{yt}}
\newcommand{\enzo}{\texttt{Enzo}}
\newcommand{\nosection}[1]{%
  \refstepcounter{section}%
  \addcontentsline{toc}{section}{\protect\numberline{\thesection}#1}%
  \markright{#1}}
\newcommand{\shellcmd}[1]{\\\indent\indent\texttt{\footnotesize\$ #1}\\}

%%%%%%end preamble

\title{Some Notes on the Weakly Nonlinear MRI Project}
\begin{document}
\maketitle

\section{Mathematical Basics}
\label{sec:basics}

Here are some really basic math stuff that might be useful to you.

\section{Computational Basics}
\label{sec:comp_basics}

We'll be making a lot of use of Python, SymPy, the iPython notebook,
and later, Dedalus 2.0. 

\subsection{Software to install}
\label{sec:software}

\begin{itemize}
\item Python3 (latest version)
\item SymPy 
\item ipython 1.0
\item matplotlib 1.3.0
\item mercurial (hg)
\end{itemize}

I know you use a Mac, which I don't, so I'm not an expert. If you
haven't already, one way to do this is to install python3, and then
install pip using the \texttt{get-pip.py} script described at
\url{http://www.pip-installer.org/en/latest/installing.html}. Once you have pip installed, you can just do 
\shellcmd{pip install sympy} 
for example, and everything should just work.

\subsection{Dedalus 2.0}
\label{sec:dedalus2}

Dedalus 2.0 isn't quite ready for us to use yet, and it's very likely
a bunch of stuff will change between now and when we are ready, but if
you're curious, it's here: \url{https://bitbucket.org/jsoishi/dedalus2}.

\subsection{Version Control with Mercurial and Bitbucket}
\label{sec:vc}

In order to facilitate our collaboration, we'll be using the mercurial
version control system. A great intro to is
\url{http://hginit.com}. 

Our shared repository will be hosted on \url{https://bitbucket.org},
starting with this very document, which you can find here:
\url{https://bitbucket.org/jsoishi/weakly_nonlinear_MRI/}



\end{document}