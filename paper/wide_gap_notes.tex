\documentclass{paper}
\newcommand{\beq}{\begin{equation}}
\newcommand{\eeq}{\end{equation}}
\newcommand{\hyd}    {{\rm H}}
\newcommand{\vobs}{v_{\rm obs}}
\newcommand{\hi}{H{\sc i}~}
\newcommand{\hii}{H{\sc ii}~}
\newcommand{\hia}{H{\sc i}}
\newcommand{\hiia}{H{\sc ii}}

\usepackage[colorlinks,urlcolor=blue,citecolor=blue,linkcolor=blue]{hyperref}
\usepackage{gensymb}
\usepackage{amsmath}
\usepackage{color}
%\usepackage{appendix}
\usepackage[titletoc,title]{appendix}
\usepackage{cleveref}


\makeatletter
\setlength{\@fptop}{0pt}
\setlength{\@fpbot}{0pt}
\setlength{\@fpsep}{0pt}
\makeatother

\newcommand{\citei}[1]{\citeauthor{#1} \citeyear{#1}}
\newcommand{\unit}[1]{\textrm{ #1}}
\newcommand{\kms}{km ${\rm s^{-1}}$}
\newcommand{\kmsa}{km ${\rm s^{-1}}$}
\newcommand{\citeia}[2]{\citeauthor{#1}~(\citeyear{#1};~#2)}

\newcommand{\uphi}{\ensuremath{u_\phi}}
\newcommand{\rhat}{\ensuremath{\mathbf{\hat{r}}}}
\newcommand{\phihat}{\ensuremath{\mathbf{\hat{\phi}}}}
\newcommand{\zhat}{\ensuremath{\mathbf{\hat{z}}}}

\newcommand\reye{\mathrm{Re}}
\newcommand\reym{\mathrm{Rm}}

\renewcommand{\topfraction}{0.85}
\renewcommand{\textfraction}{0.1}
\renewcommand{\floatpagefraction}{0.75}

\begin{document}

\title{Notes on Wide Gap}

\section{Basic Equations}
\label{sec:equations}
The so-called Stokes stream function, used in axisymmetric situations, is given by 
\begin{equation}
  \label{eq:stokes}
  \mathbf{u} \, = \, \left[\begin{matrix}
\frac{1}{r} \partial_z \psi\ \rhat\\
\uphi \ \phihat\\
-\frac{1}{r} \partial_r \psi\ \zhat\\
\end{matrix}\right];
\end{equation}
here we define $A$ in the same way. \\

Using the definitions in 

\begin{multline}
  \label{eq:psi}
\partial_t \left[ \frac{1}{r} \left(\nabla^2 \psi - \frac{2 \partial_r \psi}{r} \right) \right] + \frac{1}{r^2} J(\psi, \nabla^2 A - \frac{2 \partial_r \psi}{r}) = \frac{\partial_z A}{r^3} \left( \nabla^2 A - \frac{2 \partial_r A}{r}\right) \\
+ \frac{1}{r} J\left(A,\frac{1}{r}\left( \nabla^2 A - \frac{2 \partial_r A}{r}\right)\right) - \frac{2 B_\phi \partial_z B_\phi}{r}\\
+ \nu \left\{ \nabla^2 \left[ \frac{1}{r} \left(\nabla^2 \psi - \frac{2 \partial_r \psi}{r}\right) \right]  -\frac{1}{r^2} \left(\nabla^2 \psi - \frac{2 \partial_r \psi}{r}\right) \right\}
\end{multline}

For the expanded form of the $\Psi$ equation, Susan gets:


\begin{equation}
  \label{eq:uy}
\partial_t \uphi + \frac{J(\psi, \uphi)}{r} + \frac{\uphi \partial_z \psi}{r^2} = \frac{J(A, B_\phi)}{r} + \frac{B_\phi \partial_z A}{r^2} + \nu \left( \nabla^2 \uphi - \frac{\uphi}{r}\right)
\end{equation}

\begin{equation}
  \label{eq:A}
\partial_t A = \frac{1}{r} J(A, \psi) + \eta \left( \nabla^2 A - \frac{2 \partial_r A}{r} \right)
\end{equation}

Susan gets:
\beq
\partial_t A = \frac{1}{r} J(A, \Psi) + \frac{1}{Rm} \left[ -\frac{1}{r} \partial_r A + \partial_r^2 A + \partial_z^2 A\right]
\eeq
%Susan got:
%beq
%\partial_t A = \frac{2}{r} J(A, \Psi) + \frac{1}{Rm} \left[ \nabla^2 A - \frac{2}{r} \partial_r A \right]
%\eeq
%....will check.

\begin{multline}
  \label{eq:By} % Susan confirms correct 1/14/16
\partial_t B_\phi = \frac{1}{r} J(A, \uphi) + \frac{1}{r} J(B_\phi, \psi)\\
+ \frac{1}{r^2} B_\phi \partial_z \psi - \frac{1}{r^2} \uphi \partial_z A + \eta \left(\nabla^2 B_\phi - \frac{1}{r^2} B_\phi \right)
\end{multline}

\section{Detailed Derivation of $\Psi$ Equation}
The $\Psi$ equation, governing the x- and z-components of the velocity, is particularly tricky to derive so I will write out the steps here.

1. Find x and z components of the momentum equation, i.e.:

\beq
\partial_t u_z + \left[u \cdot \nabla u \right]_z = \left[ \left( \nabla \times B \right) \times B \right]_z + \frac{1}{\reye}\left[ \nabla^2 u \right] _z
\eeq

We sub in our stream/flux function notation and expand the operators in cylindrical coordinates. Then take $\partial_r$ of the resulting equation to obtain:

\begin{multline}
\frac{1}{r^2} \partial_t \partial_r \Psi - \frac{1}{r} \partial_t \partial_r^2 \Psi - \frac{3}{r^4}\partial_z \Psi \partial_r \Psi + \frac{1}{r^3} \partial_r \left( \partial_z \Psi \partial_r \Psi \right) + \frac{2}{r^3} \partial_z \Psi \partial_r^2 \Psi - \frac{1}{r^2} \partial_r \left(\partial_z \Psi \partial_r^2 \Psi \right)\\
- \frac{2}{r^3} \partial_r \Psi \partial_r\partial_z \Psi + \frac{1}{r^2} \partial_r \left(\partial_r \Psi \partial_r \partial_z \Psi\right) = \\
\partial_r\left(B_\theta \partial_z B_\theta\right) + \frac{2}{r^3} \partial_z^2 A \partial_z A - \frac{1}{r^2} \partial_r \left(\partial_z^2 A \partial_z A\right) + \frac{3}{r^4} \partial_z A \partial_r A - \frac{1}{r^3} \partial_r \left(\partial_z A \partial_r A \right) - \frac{2}{r^3} \partial_z A \partial_r^2 A + \frac{1}{r^2} \partial_r \left(\partial_z A \partial_r^2 A \right) \\
+ \frac{1}{\reye} \left[\frac{3}{r^4} \partial_r \Psi - \frac{3}{r^3} \partial_r^2 \Psi + \frac{2}{r^2} \partial_r^3 \psi - \frac{1}{r} \partial_r^4 \Psi\right]
\end{multline}

\section{Recovery of Narrow Gap Equations}
\label{sec:narrow_gap_recovery}

\appendix

\section{Cylindrical derivatives}
\label{sec:cylindrical_deriv}

Everything here follows \url{http://farside.ph.utexas.edu/teaching/336L/Fluidhtml/node177.html#scyl}.

For a scalar field $\psi$, 

\begin{equation}
  \label{eq:del_scalar}
  \nabla \psi = \frac{\partial \psi}{\partial r} \rhat + \frac{1}{r} \frac{\partial \psi}{\partial \phi} \phihat + \frac{\partial \psi}{\partial z} \zhat.
\end{equation}

However, for a \emph{vector} field $\mathbf{u}$,
\begin{equation}
  \label{eq:div}
  \mathbf{\nabla \cdot u} = \frac{1}{r} \frac{\partial (r u_r)}{\partial r} + \frac{1}{r} \frac{\partial \uphi}{\partial \phi}  + \frac{\partial u_z}{\partial z}
\end{equation}

and 

\begin{equation}
  \label{eq:curl}
  \mathbf{\nabla \times u} = \left( \frac{1}{r} \frac{\partial u_z}{\partial \phi} - \frac{\partial u_\phi}{\partial z}\right)\ \rhat + \left(\frac{\partial u_r}{\partial z} - \frac{\partial u_z}{\partial r}\right)\ \phihat  + \left( \frac{1}{r} \frac{\partial (r \uphi)}{\partial r} - \frac{1}{r}\frac{\partial u_r}{\partial \phi}\right)\ \zhat.
\end{equation}

We also need the $\phi$ component of the convective derivative $\mathbf{u \cdot \nabla u}$,
\begin{equation}
  \label{eq:convective_deriv_phi}
  \left[\mathbf{u \cdot \nabla u} \right]_\phi = \mathbf{u \cdot \nabla} \uphi + \frac{ u_r \uphi}{r},
\end{equation}
and finally, the vector Laplacian,
\begin{equation}
  \label{eq:vec_lap_r}
  (\nabla^2 \mathbf{u})_r = \nabla^2 u_r - \frac{u_r}{r^2} - \frac{2}{r^2} \frac{\partial \uphi}{\partial \phi}
\end{equation}

\begin{equation}
  \label{eq:vec_lap_phi}
  (\nabla^2 \mathbf{u})_\phi = \nabla^2 u_\phi + \frac{2}{r^2} \frac{\partial u_r}{\partial \phi} - \frac{u_\phi}{r^2} 
\end{equation}

\begin{equation}
  \label{eq:vec_lap_z}
  (\nabla^2 \mathbf{u})_z = \nabla^2 u_z,
\end{equation}
where $\nabla$ on the vector components is given by equation~(\ref{eq:del_scalar}).

Note that, expanding the definition of the vector Laplacian, where the cylindrical scalar Laplacian is substituted in for $\nabla^2 u_r$ and $\nabla^2 u_z$  


\end{document}
