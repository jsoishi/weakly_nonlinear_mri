\documentclass{paper}
\newcommand{\beq}{\begin{equation}}
\newcommand{\eeq}{\end{equation}}
\newcommand{\hyd}    {{\rm H}}
\newcommand{\vobs}{v_{\rm obs}}
\newcommand{\hi}{H{\sc i}~}
\newcommand{\hii}{H{\sc ii}~}
\newcommand{\hia}{H{\sc i}}
\newcommand{\hiia}{H{\sc ii}}

\usepackage[colorlinks,urlcolor=blue,citecolor=blue,linkcolor=blue]{hyperref}
\usepackage{gensymb}
\usepackage{amsmath}
\usepackage{color}
%\usepackage{appendix}
\usepackage[titletoc,title]{appendix}
\usepackage{cleveref}
\usepackage{mathtools}

\makeatletter
\setlength{\@fptop}{0pt}
\setlength{\@fpbot}{0pt}
\setlength{\@fpsep}{0pt}
\makeatother

\newcommand{\citei}[1]{\citeauthor{#1} \citeyear{#1}}
\newcommand{\unit}[1]{\textrm{ #1}}
\newcommand{\kms}{km ${\rm s^{-1}}$}
\newcommand{\kmsa}{km ${\rm s^{-1}}$}
\newcommand{\citeia}[2]{\citeauthor{#1}~(\citeyear{#1};~#2)}

\newcommand{\uphi}{\ensuremath{u_\phi}}
\newcommand{\rhat}{\ensuremath{\mathbf{\hat{r}}}}
\newcommand{\phihat}{\ensuremath{\mathbf{\hat{\phi}}}}
\newcommand{\zhat}{\ensuremath{\mathbf{\hat{z}}}}

\newcommand\reye{\mathrm{Re}}
\newcommand\reym{\mathrm{Rm}}

\renewcommand{\topfraction}{0.85}
\renewcommand{\textfraction}{0.1}
\renewcommand{\floatpagefraction}{0.75}

\begin{document}

\title{Notes on Wide Gap}

\section{Basic Equations}
\label{sec:equations}
The so-called Stokes stream function, used in axisymmetric situations, is given by 
\begin{equation}
  \label{eq:stokes}
  \mathbf{u} \, = \, \left[\begin{matrix}
\frac{1}{r} \partial_z \psi\ \rhat\\
\uphi \ \phihat\\
-\frac{1}{r} \partial_r \psi\ \zhat\\
\end{matrix}\right];
\end{equation}
here we define $A$ in the same way. \\

Using the definitions in 

\begin{multline}
  \label{eq:psi}
\partial_t \left[ \frac{1}{r} \left(\nabla^2 \psi - \frac{2 \partial_r \psi}{r} \right) \right] + \frac{1}{r^2} J(\psi, \nabla^2 A - \frac{2 \partial_r \psi}{r}) = \frac{\partial_z A}{r^3} \left( \nabla^2 A - \frac{2 \partial_r A}{r}\right) \\
+ \frac{1}{r} J\left(A,\frac{1}{r}\left( \nabla^2 A - \frac{2 \partial_r A}{r}\right)\right) - \frac{2 B_\phi \partial_z B_\phi}{r}\\
+ \nu \left\{ \nabla^2 \left[ \frac{1}{r} \left(\nabla^2 \psi - \frac{2 \partial_r \psi}{r}\right) \right]  -\frac{1}{r^2} \left(\nabla^2 \psi - \frac{2 \partial_r \psi}{r}\right) \right\}
\end{multline}

For the expanded form of the $\Psi$ equation, Susan gets:


\begin{equation}
  \label{eq:uy}
\partial_t \uphi + \frac{J(\psi, \uphi)}{r} + \frac{\uphi \partial_z \psi}{r^2} = \frac{J(A, B_\phi)}{r} + \frac{B_\phi \partial_z A}{r^2} + \nu \left( \nabla^2 \uphi - \frac{\uphi}{r}\right)
\end{equation}

%\begin{equation}
 % \label{eq:A}
%\partial_t A = \frac{1}{r} J(A, \psi) + \eta \left( \nabla^2 A - \frac{2 \partial_r A}{r} \right)
%\end{equation}

%Susan confirms correct (just switched eta -> 1/Rm)
\beq
 \label{eq:A}
\partial_t A = \frac{1}{r} J(A, \psi) + \frac{1}{\reym} \left( \nabla^2 A - \frac{2 \partial_r A}{r} \right)
%\partial_t A = \frac{1}{r} J(A, \Psi) + \frac{1}{Rm} \left[ -\frac{1}{r} \partial_r A + \partial_r^2 A + \partial_z^2 A\right]
\eeq
%Susan got:
%beq
%\partial_t A = \frac{2}{r} J(A, \Psi) + \frac{1}{Rm} \left[ \nabla^2 A - \frac{2}{r} \partial_r A \right]
%\eeq
%....will check.

\begin{multline}
  \label{eq:By} % Susan confirms correct 1/14/16
\partial_t B_\phi = \frac{1}{r} J(A, \uphi) + \frac{1}{r} J(B_\phi, \psi)\\
+ \frac{1}{r^2} B_\phi \partial_z \psi - \frac{1}{r^2} \uphi \partial_z A + \eta \left(\nabla^2 B_\phi - \frac{1}{r^2} B_\phi \right)
\end{multline}

\section{Detailed Derivation of $\Psi$ Equation}
The $\Psi$ equation, governing the x- and z-components of the velocity, is particularly tricky to derive so I will write out the steps here.

1. Find $\rhat$ and $\zhat$ components of the momentum equation, i.e.:

\beq
\partial_t u_z + \left[u \cdot \nabla u \right]_z = \left[ \left( \nabla \times B \right) \times B \right]_z + \frac{1}{\reye}\left[ \nabla^2 u \right] _z
\eeq

We sub in our stream/flux function notation and expand the operators in cylindrical coordinates. Then take $\partial_r$ of the resulting equation to obtain:

\begin{multline}
\frac{1}{r^2} \partial_t \partial_r \Psi - \frac{1}{r} \partial_t \partial_r^2 \Psi - \frac{3}{r^4}\partial_z \Psi \partial_r \Psi + \frac{1}{r^3} \partial_r \left( \partial_z \Psi \partial_r \Psi \right) + \frac{2}{r^3} \partial_z \Psi \partial_r^2 \Psi - \frac{1}{r^2} \partial_r \left(\partial_z \Psi \partial_r^2 \Psi \right)\\
- \frac{2}{r^3} \partial_r \Psi \partial_r\partial_z \Psi + \frac{1}{r^2} \partial_r \left(\partial_r \Psi \partial_r \partial_z \Psi\right) = \\
\partial_r\left(B_\phi \partial_z B_\phi\right) + \frac{2}{r^3} \partial_z^2 A \partial_z A - \frac{1}{r^2} \partial_r \left(\partial_z^2 A \partial_z A\right) + \frac{3}{r^4} \partial_z A \partial_r A - \frac{1}{r^3} \partial_r \left(\partial_z A \partial_r A \right) - \frac{2}{r^3} \partial_z A \partial_r^2 A \\
+ \frac{1}{r^2} \partial_r \left(\partial_z A \partial_r^2 A \right) 
+ \frac{1}{\reye} \left[\frac{3}{r^4} \partial_r \Psi - \frac{3}{r^3} \partial_r^2 \Psi + \frac{2}{r^2} \partial_r^3 \Psi - \frac{1}{r} \partial_r^4 \Psi\right]
\end{multline}

Repeat this process for the $\rhat$ component of the momentum equation,

\beq
\partial_t u_r + \left[u \cdot \nabla u \right]_r = \left[ \left( \nabla \times B \right) \times B \right]_r + \frac{1}{\reye}\left[ \nabla^2 u \right] _r
\eeq

and take $\partial_z$ of the expanded equation to obtain

\begin{multline}
\frac{1}{r} \partial_t \partial_z^2 \Psi - \frac{1}{r^3} \partial_z \left(\partial_z \Psi \partial_z \Psi\right) + \frac{1}{r^2} \partial_z\left(\partial_z\Psi \partial_z\partial_r \Psi\right) - \frac{1}{r^2} \partial_z \left(\partial_r \Psi \partial_z^2 \Psi\right) - \frac{1}{r} 2 u_\phi \partial_z u_\phi \\
= - \frac{1}{r^2} \partial_z^3 A \partial_r A - \frac{1}{r^2} \partial_z^2 A \partial_r \partial_z A + \frac{2}{r^3} \partial_r\partial_z A \partial_r A - \frac{1}{r^2} \partial_r^2 \partial_z A \partial_rA - \frac{1}{r^2} \partial_r^2 A \partial_r \partial_z A \\
+ \frac{1}{\reye} \left[ -\frac{1}{r^2} \partial_z^2 \partial_r \Psi + \frac{1}{r} \partial_z^2 \partial_r^2 \Psi + \frac{1}{r} \partial_z^4\Psi \right]
\end{multline}

It is clear from the $\partial_t$ terms that we must combine these equations by subtracting the $\zhat$ equation from the $\rhat$ equation.

When we do, we can simplify the LHS of the equation to:

\beq
\frac{1}{r}\partial_t \left(\nabla^2 \Psi - \frac{2}{r} \partial_r \Psi \right) + J\left(\Psi, \frac{1}{r^2} \left( \nabla^2 \Psi - \frac{2}{r} \partial_r\Psi\right) \right) - \frac{1}{r} 2 u_\phi \partial_z u_\phi
\eeq

Note that the relevant quantity appears to be $\nabla^2 \Psi - \frac{2}{r} \partial_r \Psi$, and that the $\frac{1}{r^2}$ in the second term cannot come out of the Jacobian (a point of disagreement with Jeff's equation above). Also I'm confused why Jeff's has no $u_\phi$ term. The RHS of this equation is significantly more complicated.

RHS viscous term:

\beq
\frac{1}{\reye} \left[ \nabla^2 \left(\frac{1}{r} \nabla^2 \Psi\right) - \frac{1}{r^3} \partial_r^2 \Psi - \frac{1}{r^4}\partial_r\Psi\right]
\eeq

Full $\Psi$ equation according to Susan:

\begin{multline}
\frac{1}{r}\partial_t \left(\nabla^2 \Psi - \frac{2}{r} \partial_r \Psi \right) + J\left(\Psi, \frac{1}{r^2} \left( \nabla^2 \Psi - \frac{2}{r} \partial_r\Psi\right) \right) - \frac{1}{r} 2 u_\phi \partial_z u_\phi \\
= J\left(A, \frac{1}{r^2} \left( \nabla^2 A - \frac{2}{r} \partial_rA\right) \right) - \frac{2}{r}B_\phi \partial_z B_\phi \\
+ \frac{1}{\reye} \left[ \nabla^2 \left(\frac{1}{r} \nabla^2 \Psi\right) - \frac{1}{r^3} \partial_r^2 \Psi - \frac{1}{r^4}\partial_r\Psi\right]
\end{multline}

Note that this is actually beautifully symmetric. Except the viscous term which still seems clunky.....

The derivation of the non-viscous term on the righthand side of the momentum equation ($\mathbf{J} \times \mathbf{B}$) is as follows. 

\beq
\partial_z\left(\left[\left(\nabla \times B \right) \times B \right]_r\right) - \partial_r \left(\left[\left(\nabla \times B\right) \times B \right]_z\right)
\eeq

\beq
= \partial_z\left(\left[\left(\partial_z B_r - \partial_r B_z \right) B_z - \left(\frac{1}{r}\partial_r\left(r B_\phi\right)\right) B_\phi \right] \right) - \partial_r \left(\left[\left(-\partial_z B_\phi\right)B_\phi - \left(\partial_z B_r - \partial_r B_z \right) B_r\right]\right)
\eeq

\begin{multline}
= -\frac{1}{r^2} \partial_z^3 A \partial_r A + \frac{1}{r^3} \partial_r \partial_z A \partial_r A - \frac{1}{r^2} \partial_r^2 \partial_z A \partial_r A - \frac{2}{r^3} \partial_z^2 A \partial_z A \\
+ \frac{1}{r^2} \partial_z^2 \partial_r A \partial_z A + \frac{3}{r^4} \partial_r A \partial_z A - \frac{3}{r^3} \partial_r^2 A \partial_z A + \frac{1}{r^2} \partial_r^3 A \partial_z A - \frac{2}{r} B_\phi \partial_z B_\phi
\end{multline}

This simplifies to 

\beq
J\left(A, \frac{1}{r^2} \left( \nabla^2 A - \frac{2}{r} \partial_rA\right) \right) - \frac{2}{r}B_\phi \partial_z B_\phi
\eeq

Full derivation of viscous term:

\beq
\partial_z \left(\frac{1}{\reye}\left[\nabla^2 u\right]_r\right) - \partial_r \left(\frac{1}{\reye}\left[\nabla^2 u\right]_z\right)
\eeq

\beq
= \frac{1}{\reye}\left[\partial_z \left(\nabla^2 u_r - \frac{1}{r^2} u_r\right) - \partial_r \left(\nabla^2 u_z\right)\right]
\eeq

\begin{multline}
= \frac{1}{\reye} \left[ - \frac{2}{r^2} \partial_z^2 \partial_r \Psi + \frac{2}{r} \partial_z^2 \partial_r^2 \Psi + \frac{1}{r} \partial_z^4 \Psi - \frac{3}{r^4} \partial_r \Psi + \frac{3}{r^3} \partial_r^2 \Psi - \frac{2}{r^2} \partial_r^3 \Psi + \frac{1}{r} \partial_r^4 \Psi \right]
\end{multline}


\section{Recovery of Narrow Gap Equations}
\label{sec:narrow_gap_recovery}

\section{Nondimensionalization}

The momentum equation must be nondimensionalized. Here are the definitions of all of the dimensional components: 

\begin{equation}
\begin{multlined}
\widetilde{u} = \Omega_0 r_0 \delta u \\
\widetilde{x} = \delta r_0 x \\
\widetilde{\nabla} = \frac{\nabla}{\delta r_0}\\
\widetilde{t} = \frac{t}{\Omega_0}\\ 
\widetilde{B} = B_0 B \\
\widetilde{P} = P_0 P \\
\widetilde{\rho} = \rho_0 \rho 
\end{multlined}
\end{equation}

If we define the velocity scale $v_0$ as the local sound speed scale $\left(c_{s0} \equiv \sqrt{\frac{P_0}{\rho_0}}\right)$, then $\Omega_0 r_0 \delta = \sqrt{\frac{P_0}{\rho_0}}$.

Nondimensionalizing the momentum equation and dividing by a factor of $\Omega_0^2 r_0 \delta$ yields

\beq
\partial_t \mathbf{u} + \mathbf{u}\cdot \mathbf{\nabla} \mathbf{u} = \frac{1}{4 \pi \rho} \frac{B_0^2}{\rho_0 \Omega_0^2 r_0^2 \delta^2} (\mathbf{\nabla}\times\mathbf{B})\times\mathbf{B} + \frac{1}{\reye}\nabla^2 \mathbf{u} - 2 \mathbf{\Omega} \times \mathbf{u} - \mathbf{\Omega} \times (\mathbf{\Omega} \times \mathbf{r})
\eeq

If we define

\beq
\beta \equiv \frac{P}{P_{mag}} = \frac{\rho_0 \Omega_0^2 r_0^2 8 \pi}{B_0^2}
\eeq

where $P_{mag} = \frac{B_0^2}{8 \pi}$, then the factor in front of $(\mathbf{\nabla}\times\mathbf{B})\times\mathbf{B} $ should be $\frac{2}{\beta}$. Agreed?

\section{Perturbed Equations}

We perturb the wide gap equations according to 

\beq
\mathbf{B} = B_0 \hat{z} + \mathbf{B_1},
\eeq

which is the same perturbation we used in the thin-gap construction. But wait! We perturbed the thin-gap equations according to 

\beq
\mathbf{u} = -q \Omega_0 r \hat{\phi} + \mathbf{u_1}
\eeq

but I don't think this is valid in the wide-gap case. To be sure, I'm going to perturb instead by 

\beq
\mathbf{u} = r\Omega(r)\hat{\phi}+ \mathbf{u_1}
\eeq

where $\Omega(r) = \Omega_0 \left(\frac{r}{r_0}\right)^{-q} = \Omega_0 \left(r\right)^{-q}$ in dimensionless coordinates (but keeping $\Omega_0$ to flag a rotational term...) Agreed?

We also add the Coriolis and centrifugal terms $-2 \mathbf{\Omega} \times \mathbf{u} - \mathbf{\Omega} \times \left(\mathbf{\Omega} \times \mathbf{r}\right)$ which expand as follows:

\beq
-2\Omega(r)\zhat \times \left( \mathbf{u_1} + r\Omega(r)\phihat\right) = -2\Omega(r)u_{r}\phihat + 2\Omega(r)u_\phi\rhat + 2r\Omega(r)^2\rhat
\eeq

\beq
-\Omega(r)\zhat \times \left(\Omega(r)\zhat \times r\rhat\right) = +\Omega(r)^2 r \rhat
\eeq

The $\rhat$ terms will have $\partial_z$ applied to them, which will destroy the $3\Omega(r)^2 r \rhat$ term and so the righthand side of the $\Psi$ equation will ultimately gain only the term $2 \Omega(r) \partial_z u_\phi \rhat$. The $\phihat$ equation gains the term $-\frac{2}{r} \Omega(r) \partial_z \Psi$ on the righthand side. 

The $\Psi$ equation becomes

\begin{multline}
\label{eq:Psi_perturbed}
\frac{1}{r}\partial_t \left(\nabla^2 \Psi - \frac{2}{r} \partial_r \Psi \right) + J\left(\Psi, \frac{1}{r^2} \left( \nabla^2 \Psi - \frac{2}{r} \partial_r\Psi\right) \right) - \frac{1}{r} 2 u_\phi \partial_z u_\phi - \partial_z\left(u_\phi \Omega(r) \right)\\
= \frac{2}{\beta} J\left(A, \frac{1}{r^2} \left( \nabla^2 A - \frac{2}{r} \partial_rA\right) \right) - \frac{2}{\beta} \frac{2}{r}B_\phi \partial_z B_\phi + 2 \Omega(r) \partial_z u_\phi \\
+ \frac{1}{\reye} \left[ \nabla^2 \left(\frac{1}{r} \nabla^2 \Psi\right) - \frac{1}{r^3} \partial_r^2 \Psi - \frac{1}{r^4}\partial_r\Psi\right] + \frac{2}{\beta} \frac{1}{r}B_0 \partial_z \left(\nabla^2 A - \frac{2}{r} \partial_r A\right) 
\end{multline}

Note that one of the terms gained from the base state in the $(\mathbf{u}\cdot \nabla)\mathbf{u}$ term can be combined with the term gained from the Coriolis term, but here I write them separately so we can check the equation.
Equation \ref{eq:uy} becomes

\begin{multline}
  \label{eq:uy_perturbed}
\partial_t \uphi + \frac{J(\psi, \uphi)}{r} + \frac{\uphi \partial_z \psi}{r^2} + \partial_z\Psi\left( \frac{2}{r}\Omega(r) + \partial_r \Omega(r) \right) = \\
\frac{2}{\beta} \frac{J(A, B_\phi)}{r} + \frac{2}{\beta} \frac{B_\phi \partial_z A}{r^2} + \frac{1}{\reye} \left( \nabla^2 \uphi - \frac{\uphi}{r}\right) + \frac{2}{\beta} B_0 \partial_z B_\phi -\frac{2}{r} \Omega(r) \partial_z \Psi
\end{multline}

%$\frac{2}{r} \partial_z \Psi \Omega(r) + \partial_z\Psi \partial_r \Omega(r)
%=\partial_z\Psi\left( \frac{2}{r}\Omega(r) + \partial_r \Omega(r) \right)$

%\begin{equation}
 % \label{eq:uy_perturbed}
%\partial_t \uphi + \frac{J(\psi, \uphi)}{r} + \frac{\uphi \partial_z \psi}{r^2} - \frac{2}{r}q \Omega_0 \partial_z \Psi= \frac{2}{\beta} \frac{J(A, B_\phi)}{r} + \frac{2}{\beta} \frac{B_\phi \partial_z A}{r^2} + \frac{1}{\reye} \left( \nabla^2 \uphi - \frac{\uphi}{r}\right) + \frac{2}{\beta} B_0 \partial_z B_\phi
%\end{equation}

%The only term gained in the $(\hat{r}, \hat{z})$ component of the induction equation is $(B_0 \hat{z} \cdot \nabla) \mathbf{u_1}$, and 

Equation \ref{eq:A} picks up $+\Omega(r) B_\phi$ from the term $- ({u_0}\hat{\phi} \cdot \nabla)\mathbf{B_1}$ and $-\Omega(r) B_\phi$ from the term $+(\mathbf{B_1} \cdot \nabla) \hat{\phi}$ and, so those cancel and the equation becomes

\beq
 \label{eq:A_perturbed}
\partial_t A = \frac{1}{r} J(A, \psi) + \frac{1}{\reym} \left( \nabla^2 A - \frac{2 \partial_r A}{r} \right) + B_0 \partial_z \Psi
\eeq

Note that this is perfectly analogous to the thin-gap version of this equation.

The $\hat{\phi}$ component of the induction equation, Equation \ref{eq:By}, becomes

\begin{multline}
  \label{eq:By_perturbed}
\partial_t B_\phi = \frac{1}{r} J(A, \uphi) + \frac{1}{r} J(B_\phi, \psi)\\
+ \frac{1}{r^2} B_\phi \partial_z \psi - \frac{1}{r^2} \uphi \partial_z A + \frac{1}{\reym} \left(\nabla^2 B_\phi - \frac{1}{r^2} B_\phi \right) + B_0 \partial_z u_\phi + \partial_z A \left( \frac{2}{r}\Omega(r) + \partial_r \Omega(r) \right) %- \frac{1}{r} q \Omega_0 \partial_z A
\end{multline}


\section{Matrix Formulation}

This is all pending a rigorous check of Equations \ref{eq:Psi_perturbed} - \ref{eq:By_perturbed}, which Jeff is working on, but I'll start putting this into a matrix construction.

First, the nonlinear vector, on the lefthand side of the equation, is

\beq
\mathbf{N} = \left[\begin{matrix}
J\left(\Psi, \frac{1}{r^2} \left(\nabla^2 \Psi - \frac{2}{r}\partial_r \Psi \right) \right) - \frac{2}{\beta} J \left(A, \frac{1}{r^2} \left( \nabla^2 A - \frac{2}{r} \partial_r A\right)\right) - \frac{2}{r} u_\phi \partial_z u_\phi + \frac{2}{\beta} \frac{2}{r} B_\phi \partial_z B_\phi \\
\frac{1}{r} J \left(\Psi, u_\phi\right) - \frac{1}{r} \frac{2}{\beta} J \left(A, B_\phi\right) + \frac{1}{r^2} u_\phi \partial_z \Psi - \frac{2}{\beta} \frac{1}{r^2} B_\phi \partial_z A \\
-\frac{1}{r} J \left(A, \Psi\right)\\
-\frac{1}{r} J \left(A, u_\phi\right) - \frac{1}{r} J \left(B_\phi, \Psi\right) - \frac{1}{r^2} B_\phi \partial_z \Psi + \frac{1}{r^2} u_\phi \partial_z A
\end{matrix}\right] 
\eeq

Note that these differ from the thin-gap nonlinear terms not only because of the curvature terms in the Jacobians, but also because of the additional advective terms in all but the $\mathbf{A}$ equation.

The $\partial_t$ terms are grouped together into 

\beq
\partial_t D = \partial_t \left[\begin{matrix}
\frac{1}{r}\nabla^2 - \frac{2}{r^2} \partial_r & 0 & 0 & 0\\
0 & 1 & 0 & 0\\
0 & 0 & 1 & 0\\
0 & 0 & 0 & 1
\end{matrix}\right]
\eeq

For the righthand side of the equation, I'll follow the same process that I did with the thin-gap. That is I'll first separate the matrices by physical meaning, and then regroup terms based on $z$-dependence. The RHS can be described by three matrices: one representing the rotational terms, one representing the background field terms, and one representing the viscous/resistive terms.

\begin{multline}
RHS = \left[\begin{matrix}
3 \Omega(r) \partial_z u_\phi \\
-\frac{4}{r} \Omega(r)\partial_z \Psi - \partial_r \Omega(r) \partial_z \Psi \\
0 \\
\frac{2}{r} \Omega(r) \partial_z A + \partial_r \Omega(r) \partial_z A
\end{matrix}\right]
+ 
\left[\begin{matrix}
\frac{2}{\beta} \frac{1}{r} B_0 \partial_z \left(\nabla^2 A - \frac{2}{r} \partial_r A\right)\\
\frac{2}{\beta} B_0 \partial_z B_\phi \\
B_0 \partial_z \Psi\\
B_0 \partial_z u_\phi
\end{matrix}\right]
+
\left[\begin{matrix}
\frac{1}{\reye} \left[ \nabla^2 \left(\frac{1}{r} \nabla^2 \Psi\right) - \frac{1}{r^3} \partial_r^2 \Psi - \frac{1}{r^4} \partial_r \Psi\right]\\
\frac{1}{\reye} \left(\nabla^2 u_\phi - \frac{1}{r} u_\phi\right)\\
\frac{1}{\reym} \left(\nabla^2 A - \frac{2}{r} \partial_r A\right)\\
\frac{1}{\reym} \left(\nabla^2 B_\phi - \frac{1}{r^2} B_\phi\right)
\end{matrix}\right] \\
=
\left[\begin{matrix}
0 & 3\Omega(r) \partial_z & 0 & 0 \\
-\frac{4}{r} \Omega(r) \partial_z - \partial_r \Omega(r) \partial_z & 0 & 0 & 0 \\
0 & 0 & 0 & 0 \\
0 & 0 & \frac{2}{r} \Omega(r) \partial_z + \partial_r \Omega(r) \partial_z & 0
\end{matrix}\right]
+\\
\left[\begin{matrix}
0 & 0 & \frac{2}{\beta} \frac{1}{r} B_0 \partial_z \left(\nabla^2 - \frac{2}{r} \partial_r\right) & 0\\
0 & 0 & 0 & \frac{2}{\beta} B_0 \partial_z\\
B_0 \partial_z & 0 & 0 & 0 \\
0 & B_0 \partial_z & 0 & 0
\end{matrix}\right]
+ 
\left[\begin{matrix}
\frac{1}{\reye} \left(\nabla^2\left(\frac{1}{r}\nabla^2\right) - \frac{1}{r^3}\partial_r^2 - \frac{1}{r^4}\partial_r\right)\\
\frac{1}{\reye} \left(\nabla^2  - \frac{1}{r} \right)\\
\frac{1}{\reym} \left(\nabla^2 - \frac{2}{r} \partial_r \right)\\
\frac{1}{\reym} \left(\nabla^2 - \frac{1}{r^2} \right)
\end{matrix}\right]
\end{multline}

We separate the viscous/resistive matrix out in terms of $\partial_z$ dependence.

\begin{multline}
\mathrm{viscous}\, \,\mathrm{terms}  = \\
 \left[\begin{matrix}
\frac{1}{\reye} \left(-\frac{3}{r^4} \partial_r + \frac{3}{r^3} \partial_r^2 - \frac{2}{r^2} \partial_r^3 + \frac{1}{r} \partial_r^4\right) & 0 & 0 & 0\\
0 & \frac{1}{\reye} \left(\partial_r^2 + \frac{1}{r} \partial_r - \frac{1}{r} \right) & 0 & 0 \\
0 & 0 & \frac{1}{\reym} \left(\partial_r^2 - \frac{1}{r} \partial_r\right) & 0 \\
0 & 0 & 0 & \frac{1}{\reym} \left(\partial_r^2 + \frac{1}{r}\partial_r - \frac{1}{r^2}\right)
\end{matrix}\right] \\
+
\partial_z^2 \left[\begin{matrix}
\frac{1}{\reye}\left(-\frac{2}{r^2}\partial_r + \frac{2}{r}\partial_r^2\right) & 0 & 0 & 0 \\
0 & \frac{1}{\reye} & 0 & 0 \\
0 & 0 & \frac{1}{\reym} & 0 \\
0 & 0 & 0 & \frac{1}{\reym}
\end{matrix}\right] \\
+
\partial_z^4\left[\begin{matrix}
\frac{1}{\reye}\frac{1}{r} & 0 & 0 & 0 \\
0 & 0 & 0 & 0 \\
0 & 0 & 0 & 0 \\
0 & 0 & 0 & 0 \\
\end{matrix}\right]
\end{multline}

Note that the form of each of the terms is different in the first matrix.

We also separate the terms in the $B0$ matrix by $\partial_z$ dependence:

\beq
B_0 \, \mathrm{terms} = 
\partial_z \left[\begin{matrix}
0 & 0 & \frac{2}{\beta} \frac{1}{r} B_0 \left(\partial_r^2 - \frac{1}{r}\partial_r\right) & 0 \\
0 & 0 & 0 & \frac{2}{\beta}B_0\\
B_0 & 0 & 0 & 0\\
0 & B_0 & 0 & 0
\end{matrix}\right] 
+
\partial_z^3 \left[\begin{matrix}
0 & 0 & \frac{2}{\beta} \frac{1}{r} B_0 & 0 \\
0 & 0 & 0 & 0 \\
0 & 0 & 0 & 0 \\
0 & 0 & 0 & 0 \\
\end{matrix}\right]
\eeq

We replace $B_0$ with our lower magnetic field strength $(1 - \epsilon^2)$:

\beq
B_0 \, \mathrm{terms} = 
(1 - \epsilon^2) \partial_z \left[\begin{matrix}
0 & 0 & \frac{2}{\beta} \frac{1}{r} \left(\partial_r^2 - \frac{1}{r}\partial_r\right) & 0 \\
0 & 0 & 0 & \frac{2}{\beta}\\
1 & 0 & 0 & 0\\
0 & 1 & 0 & 0
\end{matrix}\right] 
+
(1 - \epsilon^2) \partial_z^3 \left[\begin{matrix}
0 & 0 & \frac{2}{\beta} \frac{1}{r} & 0 \\
0 & 0 & 0 & 0 \\
0 & 0 & 0 & 0 \\
0 & 0 & 0 & 0 \\
\end{matrix}\right]
\eeq

We now define, in analogy to the thin-gap matrices, the following:

\beq
L_0 = \left[\begin{matrix}
\frac{1}{\reye} \left(-\frac{3}{r^4} \partial_r + \frac{3}{r^3} \partial_r^2 - \frac{2}{r^2} \partial_r^3 + \frac{1}{r} \partial_r^4\right) & 0 & 0 & 0\\
0 & \frac{1}{\reye} \left(\partial_r^2 + \frac{1}{r} \partial_r - \frac{1}{r} \right) & 0 & 0 \\
0 & 0 & \frac{1}{\reym} \left(\partial_r^2 - \frac{1}{r} \partial_r\right) & 0 \\
0 & 0 & 0 & \frac{1}{\reym} \left(\partial_r^2 + \frac{1}{r}\partial_r - \frac{1}{r^2}\right)
\end{matrix}\right]
\eeq

\beq
L_1 = \left[\begin{matrix}
0 & 3\Omega(r) & \frac{2}{\beta} \frac{1}{r} \left(\partial_r^2 - \frac{1}{r}\partial_r\right) & 0 \\
-\frac{4}{r} \Omega(r) - \partial_r \Omega(r) & 0 & 0 & \frac{2}{\beta} \\
1 & 0 & 0 & 0 \\
0 & 1 & \frac{2}{r} \Omega(r)  + \partial_r \Omega(r) & 0
\end{matrix}\right]
\eeq

%%%%%%%%%%%%%%%%%%%%%%%%%%%%%%%%%%%%%%%%%
%                                              APPENDIX                                                             %
%%%%%%%%%%%%%%%%%%%%%%%%%%%%%%%%%%%%%%%%%
\appendix

\section{Cylindrical derivatives}
\label{sec:cylindrical_deriv}

Everything here follows \url{http://farside.ph.utexas.edu/teaching/336L/Fluidhtml/node177.html#scyl}.

For a scalar field $\psi$, 

\begin{equation}
  \label{eq:del_scalar}
  \nabla \psi = \frac{\partial \psi}{\partial r} \rhat + \frac{1}{r} \frac{\partial \psi}{\partial \phi} \phihat + \frac{\partial \psi}{\partial z} \zhat.
\end{equation}

However, for a \emph{vector} field $\mathbf{u}$,
\begin{equation}
  \label{eq:div}
  \mathbf{\nabla \cdot u} = \frac{1}{r} \frac{\partial (r u_r)}{\partial r} + \frac{1}{r} \frac{\partial \uphi}{\partial \phi}  + \frac{\partial u_z}{\partial z}
\end{equation}

and 

\begin{equation}
  \label{eq:curl}
  \mathbf{\nabla \times u} = \left( \frac{1}{r} \frac{\partial u_z}{\partial \phi} - \frac{\partial u_\phi}{\partial z}\right)\ \rhat + \left(\frac{\partial u_r}{\partial z} - \frac{\partial u_z}{\partial r}\right)\ \phihat  + \left( \frac{1}{r} \frac{\partial (r \uphi)}{\partial r} - \frac{1}{r}\frac{\partial u_r}{\partial \phi}\right)\ \zhat.
\end{equation}

We also need the $\phi$ component of the convective derivative $\mathbf{u \cdot \nabla u}$,
\begin{equation}
  \label{eq:convective_deriv_phi}
  \left[\mathbf{u \cdot \nabla u} \right]_\phi = \mathbf{u \cdot \nabla} \uphi + \frac{ u_r \uphi}{r},
\end{equation}
and finally, the vector Laplacian,
\begin{equation}
  \label{eq:vec_lap_r}
  (\nabla^2 \mathbf{u})_r = \nabla^2 u_r - \frac{u_r}{r^2} - \frac{2}{r^2} \frac{\partial \uphi}{\partial \phi}
\end{equation}

\begin{equation}
  \label{eq:vec_lap_phi}
  (\nabla^2 \mathbf{u})_\phi = \nabla^2 u_\phi + \frac{2}{r^2} \frac{\partial u_r}{\partial \phi} - \frac{u_\phi}{r^2} 
\end{equation}

\begin{equation}
  \label{eq:vec_lap_z}
  (\nabla^2 \mathbf{u})_z = \nabla^2 u_z,
\end{equation}
where $\nabla$ on the vector components is given by equation~(\ref{eq:del_scalar}).

Note that, expanding the definition of the vector Laplacian, where the cylindrical scalar Laplacian is substituted in for $\nabla^2 u_r$ and $\nabla^2 u_z$  


\end{document}
