\documentclass{paper}
\newcommand{\beq}{\begin{equation}}
\newcommand{\eeq}{\end{equation}}
\newcommand{\hyd}    {{\rm H}}
\newcommand{\vobs}{v_{\rm obs}}
\newcommand{\hi}{H{\sc i}~}
\newcommand{\hii}{H{\sc ii}~}
\newcommand{\hia}{H{\sc i}}
\newcommand{\hiia}{H{\sc ii}}

\usepackage[colorlinks,urlcolor=blue,citecolor=blue,linkcolor=blue]{hyperref}
\usepackage{gensymb}
\usepackage{amsmath}
\usepackage{color}
%\usepackage{appendix}
\usepackage[titletoc,title]{appendix}
\usepackage{cleveref}
\usepackage{mathtools}

\makeatletter
\setlength{\@fptop}{0pt}
\setlength{\@fpbot}{0pt}
\setlength{\@fpsep}{0pt}
\makeatother

\newcommand{\citei}[1]{\citeauthor{#1} \citeyear{#1}}
\newcommand{\unit}[1]{\textrm{ #1}}
\newcommand{\kms}{km ${\rm s^{-1}}$}
\newcommand{\kmsa}{km ${\rm s^{-1}}$}
\newcommand{\citeia}[2]{\citeauthor{#1}~(\citeyear{#1};~#2)}

\newcommand{\uphi}{\ensuremath{u_\phi}}
\newcommand{\rhat}{\ensuremath{\mathbf{\hat{r}}}}
\newcommand{\phihat}{\ensuremath{\mathbf{\hat{\phi}}}}
\newcommand{\zhat}{\ensuremath{\mathbf{\hat{z}}}}
\newcommand{\xhat}{\ensuremath{\mathbf{\hat{x}}}}
\newcommand{\yhat}{\ensuremath{\mathbf{\hat{y}}}}

\newcommand\reye{\mathrm{Re}}
\newcommand\reym{\mathrm{Rm}}

\renewcommand{\topfraction}{0.85}
\renewcommand{\textfraction}{0.1}
\renewcommand{\floatpagefraction}{0.75}

\begin{document}

\title{Notes on Wide Gap}

\section{Base Flow}
In a viscous, Taylor-Couette device, there is only one base state that
satisfies no-slip boundary conditions and the Navier-Stokes
equations. This is called Couette flow, and can be written as
\begin{equation}
  \label{eq:couette_flow}
  \Omega(r) = a + \frac{b}{r^2},
\end{equation}
where $a = (\Omega_2 r^2_2 - \Omega_1 r^2_1)/(r^2_2 - r^2_1)$ and
$b = r^2_1 r^2_2 (\Omega_1 - \Omega_2)/(r^2_2 - r^2_1)$. Here,
$\Omega_1$ is the rotation rate at the inner cylinder radius $r_1$,
and likewise for the outer cylinder $r_2$. 

Keplerian flow is $\Omega \propto r^{-3/2}$; these appear to be
incompatible. However, in Couette flow, we can approximate Keplerian
flow by creating what is called ``quasi-Keplerian'' flow. 

Defining $q \equiv -d \ln \Omega/d \ln r$, we can see that for Couette flow,
\begin{equation}
  \label{eq:couette_q}
  q \equiv -\frac{d \ln \Omega}{d \ln r} = \frac{2 b}{a r^2 + b}.
\end{equation}
Thus, assuming $r_1$ and $r_2$ are fixed by the experiment, through
judicious choices of $\Omega_1$ and $\Omega_2$, we can make
$q(r_0) = 3/2$, where $r_0$ is some reference radius. Goodman \& Ji do
exactly this with $\bar{\zeta}$.

So, we have to choose $\Omega(r) = a + b/r^2$; everywhere it appears in our equations.

\section{Basic Equations}
\label{sec:equations}
The so-called Stokes stream function, used in axisymmetric situations, is given by 
\begin{equation}
  \label{eq:stokes}
  \mathbf{u} \, = \, \left[\begin{matrix}
\frac{1}{r} \partial_z \psi\ \rhat\\
\uphi \ \phihat\\
-\frac{1}{r} \partial_r \psi\ \zhat\\
\end{matrix}\right];
\end{equation}
here we define $A$ in the same way. \\

Using the definitions in 

\begin{multline}
  \label{eq:psi}
\partial_t \left[ \frac{1}{r} \left(\nabla^2 \psi - \frac{2 \partial_r \psi}{r} \right) \right] + \frac{1}{r^2} J(\psi, \nabla^2 A - \frac{2 \partial_r \psi}{r}) = \frac{\partial_z A}{r^3} \left( \nabla^2 A - \frac{2 \partial_r A}{r}\right) \\
+ \frac{1}{r} J\left(A,\frac{1}{r}\left( \nabla^2 A - \frac{2 \partial_r A}{r}\right)\right) - \frac{2 B_\phi \partial_z B_\phi}{r}\\
+ \nu \left\{ \nabla^2 \left[ \frac{1}{r} \left(\nabla^2 \psi - \frac{2 \partial_r \psi}{r}\right) \right]  -\frac{1}{r^2} \left(\nabla^2 \psi - \frac{2 \partial_r \psi}{r}\right) \right\}
\end{multline}

For the expanded form of the $\Psi$ equation, Susan gets:


\begin{equation}
  \label{eq:uy}
\partial_t \uphi + \frac{J(\psi, \uphi)}{r} + \frac{\uphi \partial_z \psi}{r^2} = \frac{J(A, B_\phi)}{r} + \frac{B_\phi \partial_z A}{r^2} + \nu \left( \nabla^2 \uphi - \frac{\uphi}{r}\right)
\end{equation}

%\begin{equation}
 % \label{eq:A}
%\partial_t A = \frac{1}{r} J(A, \psi) + \eta \left( \nabla^2 A - \frac{2 \partial_r A}{r} \right)
%\end{equation}

%Susan confirms correct (just switched eta -> 1/Rm)
\beq
 \label{eq:A}
\partial_t A = \frac{1}{r} J(A, \psi) + \frac{1}{\reym} \left( \nabla^2 A - \frac{2 \partial_r A}{r} \right)
%\partial_t A = \frac{1}{r} J(A, \Psi) + \frac{1}{Rm} \left[ -\frac{1}{r} \partial_r A + \partial_r^2 A + \partial_z^2 A\right]
\eeq
%Susan got:
%beq
%\partial_t A = \frac{2}{r} J(A, \Psi) + \frac{1}{Rm} \left[ \nabla^2 A - \frac{2}{r} \partial_r A \right]
%\eeq
%....will check.

\begin{multline}
  \label{eq:By} % Susan confirms correct 1/14/16
\partial_t B_\phi = \frac{1}{r} J(A, \uphi) + \frac{1}{r} J(B_\phi, \psi)\\
+ \frac{1}{r^2} B_\phi \partial_z \psi - \frac{1}{r^2} \uphi \partial_z A + \eta \left(\nabla^2 B_\phi - \frac{1}{r^2} B_\phi \right)
\end{multline}

\section{Detailed Derivation of $\Psi$ Equation}
The $\Psi$ equation, governing the x- and z-components of the velocity, is particularly tricky to derive so I will write out the steps here.

1. Find $\rhat$ and $\zhat$ components of the momentum equation, i.e.:

\beq
\partial_t u_z + \left[u \cdot \nabla u \right]_z = \left[ \left( \nabla \times B \right) \times B \right]_z + \frac{1}{\reye}\left[ \nabla^2 u \right] _z
\eeq

We sub in our stream/flux function notation and expand the operators in cylindrical coordinates. Then take $\partial_r$ of the resulting equation to obtain:

\begin{multline}
\frac{1}{r^2} \partial_t \partial_r \Psi - \frac{1}{r} \partial_t \partial_r^2 \Psi - \frac{3}{r^4}\partial_z \Psi \partial_r \Psi + \frac{1}{r^3} \partial_r \left( \partial_z \Psi \partial_r \Psi \right) + \frac{2}{r^3} \partial_z \Psi \partial_r^2 \Psi - \frac{1}{r^2} \partial_r \left(\partial_z \Psi \partial_r^2 \Psi \right)\\
- \frac{2}{r^3} \partial_r \Psi \partial_r\partial_z \Psi + \frac{1}{r^2} \partial_r \left(\partial_r \Psi \partial_r \partial_z \Psi\right) = \\
\partial_r\left(B_\phi \partial_z B_\phi\right) + \frac{2}{r^3} \partial_z^2 A \partial_z A - \frac{1}{r^2} \partial_r \left(\partial_z^2 A \partial_z A\right) + \frac{3}{r^4} \partial_z A \partial_r A - \frac{1}{r^3} \partial_r \left(\partial_z A \partial_r A \right) - \frac{2}{r^3} \partial_z A \partial_r^2 A \\
+ \frac{1}{r^2} \partial_r \left(\partial_z A \partial_r^2 A \right) 
+ \frac{1}{\reye} \left[\frac{3}{r^4} \partial_r \Psi - \frac{3}{r^3} \partial_r^2 \Psi + \frac{2}{r^2} \partial_r^3 \Psi - \frac{1}{r} \partial_r^4 \Psi\right]
\end{multline}

Repeat this process for the $\rhat$ component of the momentum equation,

\beq
\partial_t u_r + \left[u \cdot \nabla u \right]_r = \left[ \left( \nabla \times B \right) \times B \right]_r + \frac{1}{\reye}\left[ \nabla^2 u \right] _r
\eeq

and take $\partial_z$ of the expanded equation to obtain

\begin{multline}
\frac{1}{r} \partial_t \partial_z^2 \Psi - \frac{1}{r^3} \partial_z \left(\partial_z \Psi \partial_z \Psi\right) + \frac{1}{r^2} \partial_z\left(\partial_z\Psi \partial_z\partial_r \Psi\right) - \frac{1}{r^2} \partial_z \left(\partial_r \Psi \partial_z^2 \Psi\right) - \frac{1}{r} 2 u_\phi \partial_z u_\phi \\
= - \frac{1}{r^2} \partial_z^3 A \partial_r A - \frac{1}{r^2} \partial_z^2 A \partial_r \partial_z A + \frac{2}{r^3} \partial_r\partial_z A \partial_r A - \frac{1}{r^2} \partial_r^2 \partial_z A \partial_rA - \frac{1}{r^2} \partial_r^2 A \partial_r \partial_z A \\
+ \frac{1}{\reye} \left[ -\frac{1}{r^2} \partial_z^2 \partial_r \Psi + \frac{1}{r} \partial_z^2 \partial_r^2 \Psi + \frac{1}{r} \partial_z^4\Psi \right]
\end{multline}

It is clear from the $\partial_t$ terms that we must combine these equations by subtracting the $\zhat$ equation from the $\rhat$ equation.

When we do, we can simplify the LHS of the equation to:

\beq
\frac{1}{r}\partial_t \left(\nabla^2 \Psi - \frac{2}{r} \partial_r \Psi \right) + J\left(\Psi, \frac{1}{r^2} \left( \nabla^2 \Psi - \frac{2}{r} \partial_r\Psi\right) \right) - \frac{1}{r} 2 u_\phi \partial_z u_\phi
\eeq

Note that the relevant quantity appears to be $\nabla^2 \Psi - \frac{2}{r} \partial_r \Psi$, and that the $\frac{1}{r^2}$ in the second term cannot come out of the Jacobian (a point of disagreement with Jeff's equation above). Also I'm confused why Jeff's has no $u_\phi$ term. The RHS of this equation is significantly more complicated.

RHS viscous term:

\beq
\frac{1}{\reye} \left[ \nabla^2 \left(\frac{1}{r} \nabla^2 \Psi\right) - \frac{1}{r^3} \partial_r^2 \Psi - \frac{1}{r^4}\partial_r\Psi\right]
\eeq

Full $\Psi$ equation according to Susan:

\begin{multline}
\label{eq:Psi_unperturbed}
\frac{1}{r}\partial_t \left(\nabla^2 \Psi - \frac{2}{r} \partial_r \Psi \right) + J\left(\Psi, \frac{1}{r^2} \left( \nabla^2 \Psi - \frac{2}{r} \partial_r\Psi\right) \right) - \frac{1}{r} 2 u_\phi \partial_z u_\phi \\
= J\left(A, \frac{1}{r^2} \left( \nabla^2 A - \frac{2}{r} \partial_rA\right) \right) - \frac{2}{r}B_\phi \partial_z B_\phi \\
+ \frac{1}{\reye} \left[ \nabla^2 \left(\frac{1}{r} \nabla^2 \Psi\right) - \frac{1}{r^3} \partial_r^2 \Psi - \frac{1}{r^4}\partial_r\Psi\right]
\end{multline}

Note that this is actually beautifully symmetric. Except the viscous term which still seems clunky.....

The derivation of the non-viscous term on the righthand side of the momentum equation ($\mathbf{J} \times \mathbf{B}$) is as follows. 

\beq
\partial_z\left(\left[\left(\nabla \times B \right) \times B \right]_r\right) - \partial_r \left(\left[\left(\nabla \times B\right) \times B \right]_z\right)
\eeq

\beq
= \partial_z\left(\left[\left(\partial_z B_r - \partial_r B_z \right) B_z - \left(\frac{1}{r}\partial_r\left(r B_\phi\right)\right) B_\phi \right] \right) - \partial_r \left(\left[\left(-\partial_z B_\phi\right)B_\phi - \left(\partial_z B_r - \partial_r B_z \right) B_r\right]\right)
\eeq

\begin{multline}
= -\frac{1}{r^2} \partial_z^3 A \partial_r A + \frac{1}{r^3} \partial_r \partial_z A \partial_r A - \frac{1}{r^2} \partial_r^2 \partial_z A \partial_r A - \frac{2}{r^3} \partial_z^2 A \partial_z A \\
+ \frac{1}{r^2} \partial_z^2 \partial_r A \partial_z A + \frac{3}{r^4} \partial_r A \partial_z A - \frac{3}{r^3} \partial_r^2 A \partial_z A + \frac{1}{r^2} \partial_r^3 A \partial_z A - \frac{2}{r} B_\phi \partial_z B_\phi
\end{multline}

This simplifies to 

\beq
J\left(A, \frac{1}{r^2} \left( \nabla^2 A - \frac{2}{r} \partial_rA\right) \right) - \frac{2}{r}B_\phi \partial_z B_\phi
\eeq

Full derivation of viscous term:

\beq
\partial_z \left(\frac{1}{\reye}\left[\nabla^2 u\right]_r\right) - \partial_r \left(\frac{1}{\reye}\left[\nabla^2 u\right]_z\right)
\eeq

\beq
= \frac{1}{\reye}\left[\partial_z \left(\nabla^2 u_r - \frac{1}{r^2} u_r\right) - \partial_r \left(\nabla^2 u_z\right)\right]
\eeq

\begin{multline}
= \frac{1}{\reye} \left[ - \frac{2}{r^2} \partial_z^2 \partial_r \Psi + \frac{2}{r} \partial_z^2 \partial_r^2 \Psi + \frac{1}{r} \partial_z^4 \Psi - \frac{3}{r^4} \partial_r \Psi + \frac{3}{r^3} \partial_r^2 \Psi - \frac{2}{r^2} \partial_r^3 \Psi + \frac{1}{r} \partial_r^4 \Psi \right]
\end{multline}


\section{Recovery of Narrow Gap Equations}
\label{sec:narrow_gap_recovery}

In order to recover the narrow gap, we take the transformation from dimensional, cylindrical coordinates $(\widetilde{r},\widetilde{z})$ to dimensioneless quantities $(x,z)$. In order to do so, we use $\widetilde{r} = r_0 (1 + \delta x)$, where $\delta = (r_{out} - r_{in})/r_0 \to 0$. Then, we recall that we must change all derivatives $\partial_{\widetilde{r}} \psi = \partial_x \psi \partial_{\widetilde{r}} x$ and likewise for $\partial_{\widetilde{z}}$. In order that the $x$ and $z$ parts of the equations come in at the same order, we must choose $\widetilde{z} = r_0 \delta z$. Then, $\partial_{\widetilde{r}} x = \delta^{-1}$ and $\partial_{\widetilde{z}} z = \delta^{-1}$. When you finally expand out all terms, you should see that everything that is not $\partial^2_x$ and $\partial^2_z$ will be at higher order in $\delta$, and thus will go to zero. For example, just taking the scalar Laplacian of $B_\phi$, 

\begin{equation}
  \nabla^2 B_\phi = \frac{1}{r} \frac{\partial}{\partial r}\left(r \frac{\partial \psi}{\partial r} \right) + \frac{\partial^2 \psi}{\partial z^2}
\end{equation}

gives

\begin{equation}
  \frac{1}{(1 + \delta x)} \delta^{-2} \partial_x \left[(1 + \delta x) \partial_x B_\psi \right] + \delta^{-2} \partial_z^2 B_\phi,
\end{equation}
and using the binomial expansion, 

\begin{equation}
  (1 - \delta x) \delta^{-2} \left(\partial_x^2 B_\psi + \delta \partial_x B_\psi + \delta x \partial_x^2 B_\psi \right) + \delta^{-2} \partial_z^2 B_\phi.
\end{equation}

When $\delta \to 0$, this leaves

\begin{equation}
  \partial_x^2 B_\phi + \partial_z^2 B_\phi,
\end{equation}
as expected. Of course, we'd also do $B_\phi \to B_y$ at the same time.

Another operator we need is the curl $\nabla \times \mathbf{A}$,

\beq
\nabla \times \mathbf{A} = -\partial_{\widetilde{z}} A_\phi \rhat + \left(\partial_{\widetilde{z}} A_r - \partial_{\widetilde{r}} A_z \right) \phihat + \frac{1}{\widetilde{r}} \partial_{\widetilde{r}} \left( \widetilde{r} A_\phi\right) \zhat
\eeq

\beq
= - \frac{1}{\delta} \partial_z A_\phi \rhat + \left(\frac{1}{\delta} \partial_z A_r - \frac{1}{\delta} \partial_x A_z\right) \phihat + \left(1 - \delta x\right) \frac{1}{\delta} \left(\partial_x A_\phi + \delta A_\phi + \delta x \partial_x A_\phi \right) \zhat
\eeq

Taking $\lim\limits_{\delta \to 0}$ of the above and changing variables $\phi \to y$, etc., we get

\beq
\nabla \times \mathbf{A} = -\partial_z A_y \xhat + \left(\partial_z A_x - \partial_x A_z\right) \yhat + \partial_x A_y \zhat
\eeq

The $u_\phi$ equation (Equation \ref{eq:uy}) reduces as follows:
\begin{equation}
\partial_t \uphi + \frac{J(\psi, \uphi)}{r} + \frac{\uphi \partial_z \psi}{r^2} = \frac{J(A, B_\phi)}{r} + \frac{B_\phi \partial_z A}{r^2} + \nu \left( \nabla^2 \uphi - \frac{\uphi}{r}\right)
\end{equation}

\beq
\begin{split}
& \delta^{-2} \partial_t u_\phi + (1 - \delta r)\delta^{-2} J(\Psi, u_\phi) + (1 - 2\delta r)\delta^{-1} u_\phi \partial_z \Psi  = \\
& (1 - \delta r)\delta^{-2} J(A, B_\phi) + (1 - 2\delta r)\delta^{-1} B_\phi \partial_z A + \\
& \nu \left[(1 - \delta x) \delta^{-2} \left(\partial_x^2 u_\phi + \delta \partial_x u_\phi + \delta x \partial_x^2 u_\phi \right) + \delta^{-2} \partial_z^2 u_\phi \right] - \nu(1 - \delta r) u_\phi
\end{split}
\eeq

\beq
\partial_t u_\phi + J(\Psi, u_\phi) = J(A, B_\phi) + \nu(\partial_x^2 u_\phi + \partial_z^2 u_\phi)
\eeq


The $A$ equation reduces as follows. We also choose $\widetilde{t} = \delta^2 \Omega_0 t$ so that time comes in at the same order as the other dimensions...

\beq
\partial_t A = \frac{1}{r} J (A, \Psi) + \frac{1}{\reym} \left(\nabla^2 A - \frac{2}{r}\partial_r A \right)
\eeq

\beq
\delta^{-2} \partial_t A = (1 - \delta x)\delta^{-2} J(A, \Psi) + \frac{1}{\reym} \left[(1 - \delta x) \delta^{-2} \left(\partial_x^2 A + \delta\partial_x A + \delta x \partial_x^2 A \right) + \delta^{-2} \partial_z^2 A - \frac{2}{\delta}(1 - \delta x) \partial_x A \right]
\eeq

This reduces to 

\beq
\partial_t A = J(A, \Psi) + \frac{1}{\reym} \nabla^2 A,
\eeq

which is the (unperturbed) thin-gap equation, when we drop all terms that are higher order than $\delta^{-2}$. Note that this means we have to drop terms of $\mathcal{O}(\delta^{-1})$, which means that this formalism cannot be applied to the perturbed equations (see Equation \ref{eq:A_perturbed}), because the term $B_0\partial_zA$ would drop out, but shouldn't. (??)

The $B_\phi$ equation
\begin{multline}
\partial_t B_\phi = \frac{1}{r} J(A, \uphi) + \frac{1}{r} J(B_\phi, \psi)\\
+ \frac{1}{r^2} B_\phi \partial_z \psi - \frac{1}{r^2} \uphi \partial_z A + \eta \left(\nabla^2 B_\phi - \frac{1}{r^2} B_\phi \right)
\end{multline}

\beq
\begin{split}
\delta^{-2} \partial_t B_\phi  & = (1 - \delta r) \delta^{-2} J(A, u_\phi) + (1 - \delta r) \delta^{-2} J(B_\phi, \Psi) \\
& + (1 - 2 \delta r) \delta^{-1} B_\phi \partial_z \Psi - (1 - 2 \delta r) \delta^{-1} u_\phi \partial_z A + \\
& \eta\left[(1 - \delta x) \delta^{-2} \left(\partial_x^2 B_\phi + \delta \partial_x B_\phi + \delta x \partial_x^2 B_\phi \right) + \delta^{-2} \partial_z^2 B_\phi\right] - \eta (1 - 2\delta r) B_\phi
\end{split}
\eeq

\beq
\partial_t B_\phi = J(A, u_\phi) + J(B_\phi, \Psi) + \eta(\partial_x^2 B_\phi + \partial_z^2 B_\phi)
\eeq

The $\Psi$ equation (Equation \ref{eq:Psi_unperturbed}) reduces as follows

\beq
\begin{split}
& (1 - \delta x) \delta^{-2} \left[(1 - \delta x) \delta^{-2} \left(\partial_x^2 \Psi + \delta \partial_x \Psi + \delta x \partial_x^2 \Psi \right) + \delta^{-2} \partial_z^2 \Psi - 2(1 - \delta x)\delta^{-1}\partial_x \Psi \right] \\
& + \delta^{-2} J(\Psi, (1-2\delta x)\left((1 - \delta x) \delta^{-2} \left(\partial_x^2 \Psi + \delta \partial_x \Psi + \delta x \partial_x^2 \Psi \right) + \delta^{-2} \partial_z^2 \Psi - 2(1 - \delta x)\delta^{-1}\partial_x \Psi \right)) \\
& - (1 - \delta x)2\delta^{-1} u_\phi \partial_z u_\phi \\
& = \delta^{-2} J(A, (1 - 2 \delta)\left((1 - \delta x) \delta^{-2} \left(\partial_x^2 \Psi + \delta \partial_x \Psi + \delta x \partial_x^2 \Psi \right) + \delta^{-2} \partial_z^2 \Psi - 2(1 - \delta x) \delta^{-1} A \right)) \\
& - 2(1 - \delta x)\delta^{-1} B_\phi \partial_z B_\phi + \frac{1}{\reye} \left[\delta^{-2}(\partial_x^2 + \delta \partial_x + \delta x \partial_x^2 + \partial_z^2)((1 - \delta x)(\partial_x^2\Psi + \delta \partial_x\Psi + \delta x \partial_x^2\Psi + \partial_z^2\Psi))\right] \\
& + \frac{1}{\reye}\left[(1 - 3\delta x)\delta^{-2} \partial_x^2 \Psi - (1 - 4 \delta x)\delta^{-1}\partial_x \Psi \right]
\end{split}
\eeq

The viscous term is

\beq
\frac{1}{\reye} (1 - 
\eeq


\section{Nondimensionalization}

\subsection{Controlled gap width nondimensionalization}

The idea is to parameterize the gap width rather than simply nondimensionalizing lengths by a point in the center of the channel $r_0$, as we did below.

There are many ways to do this but one option is to define 

\beq
\widetilde{r} = r_0 + \widetilde{x}
\eeq 

Where tildes indicate dimensional quantities. Then for $\widetilde{x}$ to stay Cartesian, we need

\beq
\widetilde{r} - r_0 \ll r_2 - r_1 = d
\eeq

where $r_2 = r_1 = d$ is the gap width. Also define $\widetilde{x} \ll d$ and $d/r_0 << 1$ for the small gap approximation. Thus when we nondimensionalize,

\beq
r = \frac{\widetilde{r}}{r_0} = 1 + \frac{\widetilde{x}}{r_0}
\eeq

Note that we could control the gap width in terms of a small parameter $\epsilon x$ instead of a fraction $\frac{x}{r_0}$, but I believe both should work the same way.

In the example Jeff worked out before, we looked at the base velocity profile in the thin gap limit. The rotation profile is given by $\Omega(r) = \Omega_0 r^{-q}$ in dimensionless quantities, and the background flow $u = r\Omega(r)$, so in our new notation:

\beq
u = \left(1 + \frac{\widetilde{x}}{r_0}\right) \Omega_0 \left(1 + \frac{\widetilde{x}}{r_0}\right)^{-q} = \Omega_0 \left(1 + \frac{\widetilde{x}}{r_0}\right)^{1-q}  
\eeq

Taylor expanding the exp(1-q) quantity, we arrive at

\beq
u = \Omega_0 \left(1 + \frac{\widetilde{x}}{r_0}\right) - \Omega_0 q \frac{\widetilde{x}}{r_0}
\eeq

Then if we define this with respect to a rest frame $\Omega_0 \left(1 + \frac{\widetilde{x}}{r_0}\right)$ and define a dimensionless $x = \widetilde{x}/r_0$, we get the Umurhan+ base flow $u = - \Omega_0 q x$. 

This makes sense to me but I'm having trouble extending it to the actual equations. This implies, for example, that the term $\frac{1}{\eta} \nabla^2 B$, which in the $\hat{\phi}$ equation is

\beq
\left[\nabla^2 B\right]_\phi = \nabla^2 B_\phi - \frac{B_\phi}{r^2},
\eeq

should become 

\beq
\nabla^2 B_\phi - B_\phi \left(1 - \frac{2 \widetilde{x}}{r_0}\right)
\eeq

after substitution and Taylor expansion of the $r^{-2}$. But clearly there's no $- B_\phi \left(1 - \frac{2 \widetilde{x}}{r_0}\right)$ in the thin gap equation. What am I missing?? Is this not the right place to be substituting in the new definition of r??

\subsection{Typical nondimensionalization}
The momentum equation must be nondimensionalized. Here are the definitions of all of the dimensional components: 

\begin{equation}
\begin{multlined}
\widetilde{u} = \Omega_0 r_0 \delta u \\
\widetilde{x} = \delta r_0 x \\
\widetilde{\nabla} = \frac{\nabla}{\delta r_0}\\
\widetilde{t} = \frac{t}{\Omega_0}\\ 
\widetilde{B} = B_0 B \\
\widetilde{P} = P_0 P \\
\widetilde{\rho} = \rho_0 \rho 
\end{multlined}
\end{equation}

If we define the velocity scale $v_0$ as the local sound speed scale $\left(c_{s0} \equiv \sqrt{\frac{P_0}{\rho_0}}\right)$, then $\Omega_0 r_0 \delta = \sqrt{\frac{P_0}{\rho_0}}$.

Nondimensionalizing the momentum equation and dividing by a factor of $\Omega_0^2 r_0 \delta$ yields

\beq
\partial_t \mathbf{u} + \mathbf{u}\cdot \mathbf{\nabla} \mathbf{u} = \frac{1}{4 \pi \rho} \frac{B_0^2}{\rho_0 \Omega_0^2 r_0^2 \delta^2} (\mathbf{\nabla}\times\mathbf{B})\times\mathbf{B} + \frac{1}{\reye}\nabla^2 \mathbf{u} - 2 \mathbf{\Omega} \times \mathbf{u} - \mathbf{\Omega} \times (\mathbf{\Omega} \times \mathbf{r})
\eeq

If we define

\beq
\beta \equiv \frac{P}{P_{mag}} = \frac{\rho_0 \Omega_0^2 r_0^2 8 \pi}{B_0^2}
\eeq

where $P_{mag} = \frac{B_0^2}{8 \pi}$, then the factor in front of $(\mathbf{\nabla}\times\mathbf{B})\times\mathbf{B} $ should be $\frac{2}{\beta}$. Agreed?

\section{Perturbed Equations}

We perturb the wide gap equations according to 

\beq
\mathbf{B} = B_0 \hat{z} + \mathbf{B_1},
\eeq

which is the same perturbation we used in the thin-gap construction. But wait! We perturbed the thin-gap equations according to 

\beq
\mathbf{u} = -q \Omega_0 r \hat{\phi} + \mathbf{u_1}
\eeq

but I don't think this is valid in the wide-gap case. To be sure, I'm going to perturb instead by 

\beq
\mathbf{u} = r\Omega(r)\hat{\phi}+ \mathbf{u_1}
\eeq

where $\Omega(r) = \Omega_0 \left(\frac{r}{r_0}\right)^{-q} = \Omega_0 \left(r\right)^{-q}$ in dimensionless coordinates (but keeping $\Omega_0$ to flag a rotational term...) Agreed?

We also add the Coriolis and centrifugal terms $-2 \mathbf{\Omega} \times \mathbf{u} - \mathbf{\Omega} \times \left(\mathbf{\Omega} \times \mathbf{r}\right)$ which expand as follows:

\beq
-2\Omega(r)\zhat \times \left( \mathbf{u_1} + r\Omega(r)\phihat\right) = -2\Omega(r)u_{r}\phihat + 2\Omega(r)u_\phi\rhat + 2r\Omega(r)^2\rhat
\eeq

\beq
-\Omega(r)\zhat \times \left(\Omega(r)\zhat \times r\rhat\right) = +\Omega(r)^2 r \rhat
\eeq

The $\rhat$ terms will have $\partial_z$ applied to them, which will destroy the $3\Omega(r)^2 r \rhat$ term and so the righthand side of the $\Psi$ equation will ultimately gain only the term $2 \Omega(r) \partial_z u_\phi \rhat$. The $\phihat$ equation gains the term $-\frac{2}{r} \Omega(r) \partial_z \Psi$ on the righthand side. 

The $\Psi$ equation becomes

\begin{multline}
\label{eq:Psi_perturbed}
\frac{1}{r}\partial_t \left(\nabla^2 \Psi - \frac{2}{r} \partial_r \Psi \right) + J\left(\Psi, \frac{1}{r^2} \left( \nabla^2 \Psi - \frac{2}{r} \partial_r\Psi\right) \right) - \frac{1}{r} 2 u_\phi \partial_z u_\phi - \partial_z\left(u_\phi \Omega(r) \right)\\
= \frac{2}{\beta} J\left(A, \frac{1}{r^2} \left( \nabla^2 A - \frac{2}{r} \partial_rA\right) \right) - \frac{2}{\beta} \frac{2}{r}B_\phi \partial_z B_\phi + 2 \Omega(r) \partial_z u_\phi \\
+ \frac{1}{\reye} \left[ \nabla^2 \left(\frac{1}{r} \nabla^2 \Psi\right) - \frac{1}{r^3} \partial_r^2 \Psi - \frac{1}{r^4}\partial_r\Psi\right] + \frac{2}{\beta} \frac{1}{r}B_0 \partial_z \left(\nabla^2 A - \frac{2}{r} \partial_r A\right) 
\end{multline}

Note that one of the terms gained from the base state in the $(\mathbf{u}\cdot \nabla)\mathbf{u}$ term can be combined with the term gained from the Coriolis term, but here I write them separately so we can check the equation.
Equation \ref{eq:uy} becomes

\begin{multline}
  \label{eq:uy_perturbed}
\partial_t \uphi + \frac{J(\psi, \uphi)}{r} + \frac{\uphi \partial_z \psi}{r^2} + \partial_z\Psi\left( \frac{2}{r}\Omega(r) + \partial_r \Omega(r) \right) = \\
\frac{2}{\beta} \frac{J(A, B_\phi)}{r} + \frac{2}{\beta} \frac{B_\phi \partial_z A}{r^2} + \frac{1}{\reye} \left( \nabla^2 \uphi - \frac{\uphi}{r}\right) + \frac{2}{\beta} B_0 \partial_z B_\phi -\frac{2}{r} \Omega(r) \partial_z \Psi
\end{multline}

%$\frac{2}{r} \partial_z \Psi \Omega(r) + \partial_z\Psi \partial_r \Omega(r)
%=\partial_z\Psi\left( \frac{2}{r}\Omega(r) + \partial_r \Omega(r) \right)$

%\begin{equation}
 % \label{eq:uy_perturbed}
%\partial_t \uphi + \frac{J(\psi, \uphi)}{r} + \frac{\uphi \partial_z \psi}{r^2} - \frac{2}{r}q \Omega_0 \partial_z \Psi= \frac{2}{\beta} \frac{J(A, B_\phi)}{r} + \frac{2}{\beta} \frac{B_\phi \partial_z A}{r^2} + \frac{1}{\reye} \left( \nabla^2 \uphi - \frac{\uphi}{r}\right) + \frac{2}{\beta} B_0 \partial_z B_\phi
%\end{equation}

%The only term gained in the $(\hat{r}, \hat{z})$ component of the induction equation is $(B_0 \hat{z} \cdot \nabla) \mathbf{u_1}$, and 

Equation \ref{eq:A} picks up $+\Omega(r) B_\phi$ from the term $- ({u_0}\hat{\phi} \cdot \nabla)\mathbf{B_1}$ and $-\Omega(r) B_\phi$ from the term $+(\mathbf{B_1} \cdot \nabla) \hat{\phi}$ and, so those cancel and the equation becomes

\beq
 \label{eq:A_perturbed}
\partial_t A = \frac{1}{r} J(A, \psi) + \frac{1}{\reym} \left( \nabla^2 A - \frac{2 \partial_r A}{r} \right) + B_0 \partial_z \Psi
\eeq

Note that this is perfectly analogous to the thin-gap version of this equation.

The $\hat{\phi}$ component of the induction equation, Equation \ref{eq:By}, becomes

\begin{multline}
  \label{eq:By_perturbed}
\partial_t B_\phi = \frac{1}{r} J(A, \uphi) + \frac{1}{r} J(B_\phi, \psi)\\
+ \frac{1}{r^2} B_\phi \partial_z \psi - \frac{1}{r^2} \uphi \partial_z A + \frac{1}{\reym} \left(\nabla^2 B_\phi - \frac{1}{r^2} B_\phi \right) + B_0 \partial_z u_\phi + \partial_z A \left( \frac{2}{r}\Omega(r) + \partial_r \Omega(r) \right) %- \frac{1}{r} q \Omega_0 \partial_z A
\end{multline}


\section{Matrix Formulation}

This is all pending a rigorous check of Equations \ref{eq:Psi_perturbed} - \ref{eq:By_perturbed}, which Jeff is working on, but I'll start putting this into a matrix construction.

First, the nonlinear vector, on the lefthand side of the equation, is

% rigorously checked from scratch 5/11/16 -- correct.
\beq\label{eq:nonlinearvector}
\mathbf{N} = \left[\begin{matrix}
J\left(\Psi, \frac{1}{r^2} \left(\nabla^2 \Psi - \frac{2}{r}\partial_r \Psi \right) \right) - \frac{2}{\beta} J \left(A, \frac{1}{r^2} \left( \nabla^2 A - \frac{2}{r} \partial_r A\right)\right) - \frac{2}{r} u_\phi \partial_z u_\phi + \frac{2}{\beta} \frac{2}{r} B_\phi \partial_z B_\phi \\
\frac{1}{r} J \left(\Psi, u_\phi\right) - \frac{1}{r} \frac{2}{\beta} J \left(A, B_\phi\right) + \frac{1}{r^2} u_\phi \partial_z \Psi - \frac{2}{\beta} \frac{1}{r^2} B_\phi \partial_z A \\
-\frac{1}{r} J \left(A, \Psi\right)\\
-\frac{1}{r} J \left(A, u_\phi\right) - \frac{1}{r} J \left(B_\phi, \Psi\right) - \frac{1}{r^2} B_\phi \partial_z \Psi + \frac{1}{r^2} u_\phi \partial_z A
\end{matrix}\right] 
\eeq

Note that these differ from the thin-gap nonlinear terms not only because of the curvature terms in the Jacobians, but also because of the additional advective terms in all but the $\mathbf{A}$ equation.

The $\partial_t$ terms are grouped together into 

\beq
\partial_t D = \partial_t \left[\begin{matrix}
\frac{1}{r}\nabla^2 - \frac{2}{r^2} \partial_r & 0 & 0 & 0\\
0 & 1 & 0 & 0\\
0 & 0 & 1 & 0\\
0 & 0 & 0 & 1
\end{matrix}\right]
\eeq

For the righthand side of the equation, I'll follow the same process that I did with the thin-gap. That is I'll first separate the matrices by physical meaning, and then regroup terms based on $z$-dependence. The RHS can be described by three matrices: one representing the rotational terms, one representing the background field terms, and one representing the viscous/resistive terms.

\begin{multline}
RHS = \left[\begin{matrix}
3 \Omega(r) \partial_z u_\phi \\
-\frac{4}{r} \Omega(r)\partial_z \Psi - \partial_r \Omega(r) \partial_z \Psi \\
0 \\
\frac{2}{r} \Omega(r) \partial_z A + \partial_r \Omega(r) \partial_z A
\end{matrix}\right]
+ 
\left[\begin{matrix}
\frac{2}{\beta} \frac{1}{r} B_0 \partial_z \left(\nabla^2 A - \frac{2}{r} \partial_r A\right)\\
\frac{2}{\beta} B_0 \partial_z B_\phi \\
B_0 \partial_z \Psi\\
B_0 \partial_z u_\phi
\end{matrix}\right]
+
\left[\begin{matrix}
\frac{1}{\reye} \left[ \nabla^2 \left(\frac{1}{r} \nabla^2 \Psi\right) - \frac{1}{r^3} \partial_r^2 \Psi - \frac{1}{r^4} \partial_r \Psi\right]\\
\frac{1}{\reye} \left(\nabla^2 u_\phi - \frac{1}{r} u_\phi\right)\\
\frac{1}{\reym} \left(\nabla^2 A - \frac{2}{r} \partial_r A\right)\\
\frac{1}{\reym} \left(\nabla^2 B_\phi - \frac{1}{r^2} B_\phi\right)
\end{matrix}\right] \\
=
\left[\begin{matrix}
0 & 3\Omega(r) \partial_z & 0 & 0 \\
-\frac{4}{r} \Omega(r) \partial_z - \partial_r \Omega(r) \partial_z & 0 & 0 & 0 \\
0 & 0 & 0 & 0 \\
0 & 0 & \frac{2}{r} \Omega(r) \partial_z + \partial_r \Omega(r) \partial_z & 0
\end{matrix}\right]
+\\
\left[\begin{matrix}
0 & 0 & \frac{2}{\beta} \frac{1}{r} B_0 \partial_z \left(\nabla^2 - \frac{2}{r} \partial_r\right) & 0\\
0 & 0 & 0 & \frac{2}{\beta} B_0 \partial_z\\
B_0 \partial_z & 0 & 0 & 0 \\
0 & B_0 \partial_z & 0 & 0
\end{matrix}\right]
+ 
\left[\begin{matrix}
\frac{1}{\reye} \left(\nabla^2\left(\frac{1}{r}\nabla^2\right) - \frac{1}{r^3}\partial_r^2 - \frac{1}{r^4}\partial_r\right)\\
\frac{1}{\reye} \left(\nabla^2  - \frac{1}{r} \right)\\
\frac{1}{\reym} \left(\nabla^2 - \frac{2}{r} \partial_r \right)\\
\frac{1}{\reym} \left(\nabla^2 - \frac{1}{r^2} \right)
\end{matrix}\right]
\end{multline}

We separate the viscous/resistive matrix out in terms of $\partial_z$ dependence.

\begin{multline}
\mathrm{viscous}\, \,\mathrm{terms}  = \\
 \left[\begin{matrix}
\frac{1}{\reye} \left(-\frac{3}{r^4} \partial_r + \frac{3}{r^3} \partial_r^2 - \frac{2}{r^2} \partial_r^3 + \frac{1}{r} \partial_r^4\right) & 0 & 0 & 0\\
0 & \frac{1}{\reye} \left(\partial_r^2 + \frac{1}{r} \partial_r - \frac{1}{r} \right) & 0 & 0 \\
0 & 0 & \frac{1}{\reym} \left(\partial_r^2 - \frac{1}{r} \partial_r\right) & 0 \\
0 & 0 & 0 & \frac{1}{\reym} \left(\partial_r^2 + \frac{1}{r}\partial_r - \frac{1}{r^2}\right)
\end{matrix}\right] \\
+
\partial_z^2 \left[\begin{matrix}
\frac{1}{\reye}\left(-\frac{2}{r^2}\partial_r + \frac{2}{r}\partial_r^2\right) & 0 & 0 & 0 \\
0 & \frac{1}{\reye} & 0 & 0 \\
0 & 0 & \frac{1}{\reym} & 0 \\
0 & 0 & 0 & \frac{1}{\reym}
\end{matrix}\right] \\
+
\partial_z^4\left[\begin{matrix}
\frac{1}{\reye}\frac{1}{r} & 0 & 0 & 0 \\
0 & 0 & 0 & 0 \\
0 & 0 & 0 & 0 \\
0 & 0 & 0 & 0 \\
\end{matrix}\right]
\end{multline}

Note that the form of each of the terms is different in the first matrix.

We also separate the terms in the $B0$ matrix by $\partial_z$ dependence:

\beq
B_0 \, \mathrm{terms} = 
\partial_z \left[\begin{matrix}
0 & 0 & \frac{2}{\beta} \frac{1}{r} B_0 \left(\partial_r^2 - \frac{1}{r}\partial_r\right) & 0 \\
0 & 0 & 0 & \frac{2}{\beta}B_0\\
B_0 & 0 & 0 & 0\\
0 & B_0 & 0 & 0
\end{matrix}\right] 
+
\partial_z^3 \left[\begin{matrix}
0 & 0 & \frac{2}{\beta} \frac{1}{r} B_0 & 0 \\
0 & 0 & 0 & 0 \\
0 & 0 & 0 & 0 \\
0 & 0 & 0 & 0 \\
\end{matrix}\right]
\eeq

We replace $B_0$ with our lower magnetic field strength $(1 - \epsilon^2)$:

\beq
B_0 \, \mathrm{terms} = 
(1 - \epsilon^2) \partial_z \left[\begin{matrix}
0 & 0 & \frac{2}{\beta} \frac{1}{r} \left(\partial_r^2 - \frac{1}{r}\partial_r\right) & 0 \\
0 & 0 & 0 & \frac{2}{\beta}\\
1 & 0 & 0 & 0\\
0 & 1 & 0 & 0
\end{matrix}\right] 
+
(1 - \epsilon^2) \partial_z^3 \left[\begin{matrix}
0 & 0 & \frac{2}{\beta} \frac{1}{r} & 0 \\
0 & 0 & 0 & 0 \\
0 & 0 & 0 & 0 \\
0 & 0 & 0 & 0 \\
\end{matrix}\right]
\eeq

We now define, in analogy to the thin-gap matrices, the following:

\beq
\mathcal{L}_0 = \left[\begin{matrix}
\frac{1}{\reye} \left(-\frac{3}{r^4} \partial_r + \frac{3}{r^3} \partial_r^2 - \frac{2}{r^2} \partial_r^3 + \frac{1}{r} \partial_r^4\right) & 0 & 0 & 0\\
0 & \frac{1}{\reye} \left(\partial_r^2 + \frac{1}{r} \partial_r - \frac{1}{r^2} \right) & 0 & 0 \\
0 & 0 & \frac{1}{\reym} \left(\partial_r^2 - \frac{1}{r} \partial_r\right) & 0 \\
0 & 0 & 0 & \frac{1}{\reym} \left(\partial_r^2 + \frac{1}{r}\partial_r - \frac{1}{r^2}\right)
\end{matrix}\right]
\eeq

%\beq
%\mathcal{L}_1 = \left[\begin{matrix}
%0 & 3\Omega(r) & \frac{2}{\beta} \frac{1}{r} \left(\partial_r^2 - \frac{1}{r}\partial_r\right) & 0 \\
%-\frac{4}{r} \Omega(r) - \partial_r \Omega(r) & 0 & 0 & \frac{2}{\beta} \\
%1 & 0 & 0 & 0 \\
%0 & 1 & \frac{2}{r} \Omega(r)  + \partial_r \Omega(r) & 0
%\end{matrix}\right]
%\eeq

% work in inertial frame instead
\beq
\mathcal{L}_1 = \left[\begin{matrix}
0 & \frac{2}{r} u_0 & \frac{2}{\beta} \frac{1}{r} \left(\partial_r^2 - \frac{1}{r}\partial_r\right) & 0 \\
(-\frac{1}{r^2} u_0 - \frac{1}{r} \partial_r u_0) & 0 & 0 & \frac{2}{\beta} \\
1 & 0 & 0 & 0 \\
0 & 1 & (\frac{1}{r}\partial_r u_0 - \frac{1}{r^2} u_0) & 0
\end{matrix}\right]
\eeq

\beq
\mathcal{L}_2 = \left[\begin{matrix}
\frac{1}{\reye}\left(-\frac{2}{r^2}\partial_r + \frac{2}{r}\partial_r^2\right) & 0 & 0 & 0 \\
0 & \frac{1}{\reye} & 0 & 0 \\
0 & 0 & \frac{1}{\reym} & 0 \\
0 & 0 & 0 & \frac{1}{\reym}
\end{matrix}\right]
\eeq

\beq
\mathcal{L}_3 = \left[\begin{matrix}
0 & 0 & \frac{2}{\beta} \frac{1}{r} & 0 \\
0 & 0 & 0 & 0 \\
0 & 0 & 0 & 0 \\
0 & 0 & 0 & 0 \\
\end{matrix}\right]
\eeq

\beq
\mathcal{L}_4 = \left[\begin{matrix}
\frac{1}{\reye}\frac{1}{r} & 0 & 0 & 0 \\
0 & 0 & 0 & 0 \\
0 & 0 & 0 & 0 \\
0 & 0 & 0 & 0 \\
\end{matrix}\right]
\eeq

\beq
\mathcal{G} = \left[\begin{matrix}
0 & 0 & \frac{2}{\beta} \frac{1}{r} \left(\partial_r^2 - \frac{1}{r}\partial_r\right) & 0 \\
0 & 0 & 0 & \frac{2}{\beta}\\
1 & 0 & 0 & 0\\
0 & 1 & 0 & 0
\end{matrix}\right] 
\eeq

Defining 

\beq
\mathcal{L}  = \mathcal{L}_0 + \mathcal{L}_1\partial_z + \mathcal{L}_2\partial_z^2 + \mathcal{L}_3\partial_z^3 + \mathcal{L}_4\partial_z^4
\eeq

and

\beq
\mathcal{\widetilde{G}} = - \mathcal{G} \partial_z - \mathcal{L}_3 \partial_z^3, 
\eeq

we have 

\beq
\mathcal{D} \partial_t \mathbf{V} + \mathbf{N} = \mathcal{L}\mathbf{V} + \epsilon^2 \mathcal{\widetilde{G}} \mathbf{V}
\eeq

which we then expand:

\beq
\begin{split}
 \partial_t & \rightarrow \epsilon^2 \partial_T \\
 \partial_z & \rightarrow \partial_z + \epsilon \partial_Z
\end{split}
\eeq

and perturb:

\beq
\mathbf{V} = \epsilon\mathbf{V}_1 + \epsilon^2 \mathbf{V}_2 + \epsilon^3 \mathbf{V}_3
\eeq

Note that as we have defined these matrices, the matrices are of course different than in the thin-gap case, and the nonlinear terms are different, but the equations for behavior at each order in $\epsilon$ are identical, that is

\beq
\mathcal{L} \mathbf{V}_1 = 0
\eeq

\beq
\mathcal{L} \mathbf{V}_2 = \mathbf{N}_2 - \mathcal{\widetilde{L}}_1 \partial_Z \mathbf{V}_1
\eeq

\beq
\mathcal{D}\partial_T\mathbf{V}_1 + \mathbf{N}_3 = \mathcal{L}\mathbf{V}_3 + \mathcal{\widetilde{L}}_1 \partial_Z \mathbf{V}_2 + \mathcal{\widetilde{L}}_2\partial_Z^2 \mathbf{V}_1 + \mathcal{\widetilde{G}}\mathbf{V}_1
\eeq

are still the correct equations to orders $\epsilon$, $\epsilon^2$, and $\epsilon^3$, respectively.


\section{Expansion of Nonlinear Terms}

Expanding the terms in Equation \ref{eq:nonlinearvector}, we find.

\beq
\mathbf{N} = \epsilon^2 \mathbf{N}_2 + \epsilon^3 \mathbf{N}_3
\eeq

\beq
\begin{split}
N_2^{\Psi} & = J\left(\Psi_1, \frac{1}{r^2} \nabla^2 \Psi_1\right) + J\left(\Psi_1, -\frac{2}{r^3}\partial_r\Psi_1\right) \\
&- \frac{2}{\beta} J \left(A_1, \frac{1}{r^2} \nabla^2 A_1\right) - \frac{2}{\beta} J\left(A_1, -\frac{2}{r^3} \partial_r A_1\right) - \frac{2}{r} u_1 \partial_z u_1 + \frac{2}{\beta} \frac{2}{r} B_1 \partial_z B_1
\end{split}
\eeq

\beq
\begin{split}
N_3^{\Psi} & = J\left(\Psi_1, \frac{1}{r^2} \nabla^2 \Psi_2\right) + J\left(\Psi_2, \frac{1}{r^2} \nabla^2\Psi_1\right) + 2 J \left(\Psi_1, \frac{1}{r^2}\partial_Z\partial_z \Psi_1\right) \\
&+ J\left(\Psi_1, -\frac{2}{r^3}\partial_r \Psi_2\right) + J\left(\Psi_2, -\frac{2}{r^3}\partial_r \Psi_1\right) + \widetilde{J}\left(\Psi_1, \frac{1}{r^2} \nabla^2 \Psi_1\right) + \widetilde{J} \left(\Psi_1, -\frac{2}{r^3}\partial_r \Psi_1\right)\\
& - \frac{2}{\beta} J\left(A_1, \frac{1}{r^2}\nabla^2 A_2\right) - \frac{2}{\beta} J\left(A_2, \frac{1}{r^2}\nabla^2 A_1\right) - \frac{4}{\beta} J\left(A_1, \frac{1}{r^2}\partial_Z\partial_z A_1\right) - \frac{2}{\beta} J\left(A_1, -\frac{2}{r^3} \partial_r A_2 \right) \\
& - \frac{2}{\beta} J\left(A_2, -\frac{2}{r^3} \partial_r A_1\right) - \frac{2}{\beta} \widetilde{J} \left(A_1, \frac{1}{r^2} \nabla^2 A_1\right) - \frac{2}{\beta} \widetilde{J} \left(A_1, -\frac{2}{r^3} \partial_r A_1\right) \\
& - \frac{2}{r} u_1 \partial_z u_2 - \frac{2}{r} u_2 \partial_z u_1 - \frac{2}{r} u_1 \partial_Z u_1 + \frac{2}{\beta}\frac{2}{r} B_1\partial_z B_2 + \frac{2}{\beta}\frac{2}{r} B_2 \partial_z B_1 + \frac{2}{\beta} \frac{2}{r} B_1 \partial_Z B_1
\end{split}
\eeq

\beq
\begin{split}
N_2^{u} = \frac{1}{r} J\left(\Psi_1, u_1\right) - \frac{1}{r} \frac{2}{\beta} J\left(A_1, B_1\right) + \frac{1}{r^2} u_1 \partial_z \Psi_1 - \frac{2}{\beta}\frac{1}{r^2} B_1 \partial_z A_1
\end{split}
\eeq

\beq
\begin{split}
N_3^u & = \frac{1}{r}J\left(\Psi_1, u_2\right) + \frac{1}{r}J\left(\Psi_2, u_1\right) + \frac{1}{r}\widetilde{J} \left(\Psi_1, u_1\right) \\
& - \frac{1}{r}\frac{2}{\beta} J\left(A_1, B_2\right) - \frac{1}{r} \frac{2}{\beta} J\left(A_2, B_1\right) - \frac{1}{r}\frac{2}{\beta}\widetilde{J}\left(A_1, B_1\right) \\
& + \frac{1}{r^2} u_1\partial_z \Psi_2 + \frac{1}{r^2} u_2 \partial_z \Psi_1 + \frac{1}{r^2} u_1 \partial_Z \Psi_1 \\
& - \frac{2}{\beta} \frac{1}{r^2} B_1 \partial_z A_2 - \frac{2}{\beta} \frac{1}{r^2} B_2 \partial_z A_1 - \frac{2}{\beta} \frac{1}{r^2} B_1 \partial_Z A_1
\end{split}
\eeq

\beq
N_2^A = -\frac{1}{r} J\left(A_1, \Psi_1\right)
\eeq

\beq
N_3^A = -\frac{1}{r} J\left(A_1, \Psi_2\right) - \frac{1}{r}J\left(A_2, \Psi_1\right) - \frac{1}{r} \widetilde{J}\left(A_1, \Psi_1\right)
\eeq

\beq
N_2^B = -\frac{1}{r} J\left(A_1, u_1\right) - \frac{1}{r} J\left(B_1, \Psi_1\right) - \frac{1}{r^2} B_1 \partial_z \Psi_1 + \frac{1}{r^2} u_1 \partial_z A_1
\eeq

\beq
\begin{split}
N_3^B & = - \frac{1}{r} J\left(A_1, u_2\right) - \frac{1}{r} J\left(A_2, u_1\right) - \frac{1}{r}\widetilde{J}\left(A_1, u_1\right) \\
& - \frac{1}{r} J\left(B_1, \Psi_2\right) - \frac{1}{r} J\left(B_2, \Psi_1\right) - \frac{1}{r} \widetilde{J} \left(B_1, u_1\right) \\
& - \frac{1}{r^2} B_1\partial_z \Psi_2 - \frac{1}{r^2} B_2 \partial_z \Psi_1 - \frac{1}{r^2} B_1 \partial_Z \Psi_1 \\
& + \frac{1}{r^2} u_1 \partial_z A_2 + \frac{1}{r^2} u_2 \partial_z A_1 + \frac{1}{r^2} u_1 \partial_Z A_1
\end{split}
\eeq

\section{Boundary Conditions}

We use boundary conditions that are no-slip and perfectly conducting. No-slip means that all components of velocity are zero at the boundary, and perfectly conducting means that the radial component of the magnetic field and the axial component of the current are both zero at the boundary. 

No-slip means $u_r = u_\phi = u_z = 0$ and therefore $u_z = - \frac{1}{r}\partial_r \Psi = 0$ and $u_r = \frac{1}{r}\partial_z \Psi = 0$. 

Perfectly conducting means $J_z = (\nabla \times \mathbf{B})_z = B_\phi + r\partial_r B_\phi = 0$

and also that $B_r = \frac{1}{r}\partial_z A = 0$.

Thus our final boundary conditions are $\Psi = u = A = \partial_r \Psi = \partial_r(r B_\phi) = 0$, for a total of ten boundary condition equations.

\section{Notes for paper}

-- Comment on channel mode formation w.r.t. parasitic modes?

Discussion:

%-- We locate threshold in the $\reym$ - $k_z$ plane. In the wide-gap case, the growth rate depends much more sensitively on $k_z$ than on $\reym$, while in the thin-gap case the opposite is true. We attribute this to the decreased influence of the boundary layers in the wide-gap geometry. (maybe?)

%The complex nature of the MRI means its properties are best parsed in idealized regimes or in simplified geometries. Many studies take 

The purely conducting boundary condition states that the axial component of the current ($\mathbf{J}_z = [\nabla \times \mathbf{B}]_z)$ must be zero at the walls. In the thin gap geometry, the purely conducting boundary condition on the azimuthal magnetic field is $\partial_x(B_y) = 0$ for axisymmetric perturbations. A spatially constant azimuthal field satisfies both the thin-gap MRI equations and this boundary condition. This neutral mode is formally included in the analysis of Umurhan et al 2007 and yields a second amplitude equation in the form of a simple diffusion equation. This amplitude equation decouples from the GLE because of the translational symmetry of the thin-gap geometry. Because that symmetry is not preserved in the wide-gap case, Umurhan et al postulate that slow variation in the wide-gap geometry will be governed by two coupled amplitude equations. However, the wide-gap conducting boundary condition $J_z = \frac{1}{r} \partial_r (r B_\phi) = 0$, together with the purely geometric term in Equation \ref{eq:By_perturbed}, conspire to prevent the wide-gap geometry from sustaining a neutral mode. We note that this mode would exist in a resistance-free approximation.

The nonlinear terms in the wide-gap case differ from their thin-gap analogues mainly in the addition of vertical advective terms. These terms derive from the advective derivatives in the momentum and induction equations, but are filtered out in the thin-gap approximation. These terms have several interesting consequences. First, they allow the nonlinear contributions at both second and third order (i.e. N$_2$ and N$_3$) not to individually satisfy the boundary conditions on $\Psi$ and $u$. Second, the advection terms break the alternating parity exhibited by the thin-gap nonlinear terms.

%%%%%%%%%%%%%%%%%%%%%%%%%%%%%%%%%%%%%%%%%
%                                              APPENDIX                                                             %
%%%%%%%%%%%%%%%%%%%%%%%%%%%%%%%%%%%%%%%%%
\appendix

\section{Cylindrical derivatives}
\label{sec:cylindrical_deriv}

Everything here follows \url{http://farside.ph.utexas.edu/teaching/336L/Fluidhtml/node257.html}.

For a scalar field $\psi$, 

\begin{equation}
  \label{eq:del_scalar}
  \nabla \psi = \frac{\partial \psi}{\partial r} \rhat + \frac{1}{r} \frac{\partial \psi}{\partial \phi} \phihat + \frac{\partial \psi}{\partial z} \zhat,
\end{equation}

and

\begin{equation}
  \label{eq:del2_scalar}
  \nabla^2 \psi = \frac{1}{r} \frac{\partial}{\partial r}\left(r \frac{\partial \psi}{\partial r} \right) + \frac{1}{r^2} \frac{\partial^2 \psi}{\partial \phi^2} + \frac{\partial^2 \psi}{\partial z^2},
\end{equation}

However, for a \emph{vector} field $\mathbf{u}$,
\begin{equation}
  \label{eq:div}
  \mathbf{\nabla \cdot u} = \frac{1}{r} \frac{\partial (r u_r)}{\partial r} + \frac{1}{r} \frac{\partial \uphi}{\partial \phi}  + \frac{\partial u_z}{\partial z}
\end{equation}

and 

\begin{equation}
  \label{eq:curl}
  \mathbf{\nabla \times u} = \left( \frac{1}{r} \frac{\partial u_z}{\partial \phi} - \frac{\partial u_\phi}{\partial z}\right)\ \rhat + \left(\frac{\partial u_r}{\partial z} - \frac{\partial u_z}{\partial r}\right)\ \phihat  + \left( \frac{1}{r} \frac{\partial (r \uphi)}{\partial r} - \frac{1}{r}\frac{\partial u_r}{\partial \phi}\right)\ \zhat.
\end{equation}

We also need the $\phi$ component of the convective derivative $\mathbf{u \cdot \nabla u}$,
\begin{equation}
  \label{eq:convective_deriv_phi}
  \left[\mathbf{u \cdot \nabla u} \right]_\phi = \mathbf{u \cdot \nabla} \uphi + \frac{ u_r \uphi}{r},
\end{equation}
and finally, the vector Laplacian,
\begin{equation}
  \label{eq:vec_lap_r}
  (\nabla^2 \mathbf{u})_r = \nabla^2 u_r - \frac{u_r}{r^2} - \frac{2}{r^2} \frac{\partial \uphi}{\partial \phi}
\end{equation}

\begin{equation}
  \label{eq:vec_lap_phi}
  (\nabla^2 \mathbf{u})_\phi = \nabla^2 u_\phi + \frac{2}{r^2} \frac{\partial u_r}{\partial \phi} - \frac{u_\phi}{r^2} 
\end{equation}

\begin{equation}
  \label{eq:vec_lap_z}
  (\nabla^2 \mathbf{u})_z = \nabla^2 u_z,
\end{equation}
where $\nabla$ on the vector components is given by equation~(\ref{eq:del_scalar}).

Note that, expanding the definition of the vector Laplacian, where the cylindrical scalar Laplacian is substituted in for $\nabla^2 u_r$ and $\nabla^2 u_z$  


\section{Figures}

\begin{figure}[h!]
\centering
\includegraphics[width=\textwidth]{../../weakly_nonlinear_mri/python/widegap/figures/u_near_threshold_test_critical}
\caption{Wide-gap first-order velocity perturbations near threshold. Unnormalized. This is with Pm = 1E-3, Rm = 0.845, Q = 0.006.
\label{fig:firstorder_u}}
\end{figure}

\begin{figure}[h!]
\centering
\includegraphics[width=\textwidth]{../../weakly_nonlinear_mri/python/widegap/figures/B_near_threshold_test_critical}
\caption{Wide-gap first-order B-field perturbations near threshold. Unnormalized. This is with Pm = 1E-3, Rm = 0.845, Q = 0.006.
\label{fig:firstorder_B}}
\end{figure}

\begin{figure}[h!]
\centering
\includegraphics[width=\textwidth]{../../weakly_nonlinear_mri/python/widegap/figures/firstorderperturbations_test}
\caption{First order perturbations, Pm = 1E-3, Rm = 0.845, Q = 0.006.
\label{fig:firstorder_all}}
\end{figure}

\begin{figure}[h!]
\centering
\includegraphics[width=\textwidth]{../../weakly_nonlinear_mri/python/widegap/figures/N2_test}
\caption{N2 from unnormalized lower-order components. This is with Pm = 1E-3, Rm = 0.845, Q = 0.006. Black = real, teal = imaginary. Why is psi not zero at the boundaries? Assuming it's because of the additional B advective term, not present in the thin-gap.
\label{fig:N2}}
\end{figure}

\begin{figure}[h!]
\centering
\includegraphics[width=\textwidth]{../../weakly_nonlinear_mri/python/widegap/figures/n3_test}
\caption{N31 from unnormalized lower-order components. This is with Pm = 1E-3, Rm = 0.845, Q = 0.006. Black = real, teal = imaginary. 
\label{fig:N31}}
\end{figure}

\end{document}
